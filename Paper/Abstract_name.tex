\documentclass[12pt]{article}

\usepackage{amssymb,amsmath,amsfonts,eurosym,geometry,
ulem,graphicx,color,setspace,sectsty,comment,footmisc,
natbib,
pdflscape,subfigure,array,hyperref, booktabs,
threeparttable, siunitx, adjustbox,
rotating}

%https://www.stefaanlippens.net/latex-trick-customizing-captions.html
\usepackage[font=sf, labelfont={sf,bf}, margin=1cm, justification=raggedright,singlelinecheck=false]{caption}

\usepackage[utf8]{inputenc}

%Change spacing between section
\usepackage{titlesec}

\titlespacing*{\section}
{0pt}{5.5ex plus 1ex minus .2ex}{4.3ex plus .2ex}
\titlespacing*{\subsection}
{0pt}{5.5ex plus 1ex minus .2ex}{4.3ex plus .2ex}

\graphicspath{{Paper/images/}}

%https://www.overleaf.com/learn/latex/Natbib_citation_styles
%https://www.overleaf.com/learn/latex/Bibtex%20bibliography%20styles#Natbib_styles
%
%\usepackage[authordate, natbib,backend=biber]{biblatex-chicago}
%\bibliography{Bibliography/SBC_bibliography.bib}
%https://gking.harvard.edu/files/natnotes2.pdf
\bibliographystyle{chicago}


%%Resize figure and make sure not one page
% https://www.overleaf.com/learn/latex/Questions/How_can_I_get_my_table_or_figure_to_stay_where_they_are,_instead_of_going_to_the_next_page%3F

\normalem

\geometry{left=1.0in,right=1.0in,top=1.0in,bottom=1.0in}
%https://tex.stackexchange.com/questions/410173/package-inputenc-error-unicode-char-u2212-error?noredirect=1&lq=1
\DeclareUnicodeCharacter{2212}{-}

\begin{document}

\begin{titlepage}


\title{New Evidence on the Chinese Environmental Policy Effectiveness in Private versus SOEs\thanks{We would like to thank the editor, anonymous reviewers for their helpful comments. We are also very greatful to Zhao Ruili and Zhou Ling for their precious help in collecting the data from the China Environment Statistics Yearbook.}}
\author{
Mathilde Maurel\thanks{CNRS, France and Centre d'Economie de la Sorbonne, Université Paris 1 Panthéon-Sorbonne, France} 
\and Thomas Pernet\thanks{Centre d'Economie de la Sorbonne, Université Paris 1 Panthéon-Sorbonne, France,
\href{mailto:t.pernetcoudrier@gmail.com}{email: t.pernetcoudrier@gmail.com} 
%email: \href{t.pernetcoudrier@gmail.com}
}
}


\date{}

\maketitle
\begin{abstract}
\noindent This paper analyses the efficiency of a set of environmental measures introduced by the 11th Five Years Plan (FYP) in China in 2006, using a rich and unique dataset borrowed from the Ministry of Environmental Protection (MEP) and the State Environmental Protection Agency (SEPA). The environmental regulation is rooted in two dimensions: the establishment of a list of TCZ (Two Control Zones) cities as soon as in 1998, which have the responsibility of reducing the emission of SO2, and the introduction in 2006 of an environmental target-based evaluation system, which is designed at the provincial level and enforced locally. By exploiting plausibly exogenous variation in regulatory stringency generated by the targets' system in China across provinces in 2006, we find evidence that pollution-intensive firms substantially decreased the emission of SO2, whereas SOEs (State Owned Enterprises) did not. We interpret these results as pointing to the evidence of a still ongoing SBC (Soft Budget Constraints) surrounding Chinese SOEs. 

The findings are robust to the inclusion of different specifications of fixed effects, and other key determinants of firm pollution. Moreover, we investigate what are the main factors behind the no-compliance of SOEs to the regulations: the location (or not) in TCZ (SPZ, Coastal) cities where the environmental (growth) policies are prioritized, the existence of turning points below (above) which growth and decrease in pollution substitute (complement) each other, the size and degree of industrial concentration which determine the possibility for firms to negotiate with the local authorities, and finally the regulation-induced adoption of cleaner technologies among polluting firms, which enhance productivity and decrease the emission of SO2.  \\
\vspace{0em}\\
\noindent\textbf{Keywords:} Environmental regulation, China, Kornai, Soft Budget Constraint, Difference-in-Difference estimation\\
\vspace{0em}\\
\noindent\textbf{JEL Codes:} Q53,Q56,P2,R11
\\

\bigskip
\end{abstract}
\setcounter{page}{0}
\thispagestyle{empty}
\end{titlepage}

\end{document}