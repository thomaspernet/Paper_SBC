\documentclass[12pt]{article}

\usepackage{amssymb,amsmath,amsfonts,eurosym,geometry,
ulem,graphicx,color,setspace,sectsty,comment,footmisc,
natbib,
pdflscape,subfigure,array,hyperref, booktabs,
threeparttable, siunitx, adjustbox,
rotating}

%https://www.stefaanlippens.net/latex-trick-customizing-captions.html
\usepackage[font=sf, labelfont={sf,bf}, margin=1cm, justification=raggedright,singlelinecheck=false]{caption}

\usepackage[utf8]{inputenc}

%Change spacing between section
\usepackage{titlesec}

\titlespacing*{\section}
{0pt}{5.5ex plus 1ex minus .2ex}{4.3ex plus .2ex}
\titlespacing*{\subsection}
{0pt}{5.5ex plus 1ex minus .2ex}{4.3ex plus .2ex}

\graphicspath{{Paper/images/}}

%https://www.overleaf.com/learn/latex/Natbib_citation_styles
%https://www.overleaf.com/learn/latex/Bibtex%20bibliography%20styles#Natbib_styles
%
%\usepackage[authordate, natbib,backend=biber]{biblatex-chicago}
%\bibliography{Bibliography/SBC_bibliography.bib}
%https://gking.harvard.edu/files/natnotes2.pdf
\bibliographystyle{chicago}


%%Resize figure and make sure not one page
% https://www.overleaf.com/learn/latex/Questions/How_can_I_get_my_table_or_figure_to_stay_where_they_are,_instead_of_going_to_the_next_page%3F

\normalem

\geometry{left=1.0in,right=1.0in,top=1.0in,bottom=1.0in}
%https://tex.stackexchange.com/questions/410173/package-inputenc-error-unicode-char-u2212-error?noredirect=1&lq=1
\DeclareUnicodeCharacter{2212}{-}

\begin{document}

\begin{titlepage}


\title{New Evidence on the Chinese Environmental Policy Effectiveness in Private versus SOEs\thanks{We would like to thank the editor, anonymous reviewers for their helpful comments. We are also very greatful to Zhao Ruili and Zhou Ling for their precious help in collecting the data from the China Environment Statistics Yearbook.}}
\author{
Mathilde Maurel\thanks{CNRS, France and Centre d'Economie de la Sorbonne, Université Paris 1 Panthéon-Sorbonne, France} 
\and Thomas Pernet\thanks{Centre d'Economie de la Sorbonne, Université Paris 1 Panthéon-Sorbonne, France,
\href{mailto:t.pernetcoudrier@gmail.com}{email: t.pernetcoudrier@gmail.com} 
%email: \href{t.pernetcoudrier@gmail.com}
}
}

\date{}

\maketitle
\begin{abstract}
\noindent This paper analyses the efficiency of a set of environmental measures introduced by the 11th Five Years Plan (FYP) in China in 2006, using a rich and unique dataset borrowed from the Ministry of Environmental Protection (MEP) and the State Environmental Protection Agency (SEPA). The environmental regulation is rooted in two dimensions: the establishment of a list of TCZ (Two Control Zones) cities as soon as in 1998, which have the responsibility of reducing the emission of SO2, and the introduction in 2006 of an environmental target-based evaluation system, which is designed at the provincial level and enforced locally. By exploiting plausibly exogenous variation in regulatory stringency generated by the targets' system in China across provinces in 2006, we find evidence that pollution-intensive firms substantially decreased the emission of SO2, whereas SOEs (State Owned Enterprises) did not. We interpret these results as pointing to the evidence of a still ongoing SBC (Soft Budget Constraints) surrounding Chinese SOEs. 

The findings are robust to the inclusion of different specifications of fixed effects, and other key determinants of firm pollution. Moreover, we investigate what are the main factors behind the no-compliance of SOEs to the regulations: the location (or not) in TCZ (SPZ, Coastal) cities where the environmental (growth) policies are prioritized, the existence of turning points below (above) which growth and decrease in pollution substitute (complement) each other, the size and degree of industrial concentration which determine the possibility for firms to negotiate with the local authorities, and finally the regulation-induced adoption of cleaner technologies among polluting firms, which enhance productivity and decrease the emission of SO2.  \\
\vspace{0em}\\
\noindent\textbf{Keywords:} Environmental regulation, China, Kornai, Soft Budget Constraint, Difference-in-Difference estimation\\
\vspace{0em}\\
\noindent\textbf{JEL Codes:} Q53,Q56,P2,R11
\\

\bigskip
\end{abstract}
\setcounter{page}{0}
\thispagestyle{empty}
\end{titlepage}
\pagebreak \newpage

\doublespacing

\section{Introduction }
\addcontentsline{toc}{section}{Introduction }

The case of China offers an illustration of the dilemma many nations face between the objectives concerning development and poverty reduction on the one hand and those of fighting against pollution on the other. Environmental protection is often at odds with poverty reduction, as the steps required to reduce poverty may entail a cost in terms of pollution to build infrastructure and stimulate economies. This tradeoff holds even more in the case of China, with provinces at very different stages of economic development, as emphasized in \citet[p.196 and p.198]{Kahn2016-fi}, as follows: $``$My boss [the provincial governor] $``$$ \ldots $ $"$  knows it is hard to kill two birds [economic growth and a clean environment] with one stone in my city for now$ \ldots $ $"$, or provinces using the strategy of $``$ridding the cage of old birds (polluting firms) in favor of new ones (clean and high-skilled firms)$"$.



This metaphor evokes the two objectives which the central government is pursuing since 2000, whereas before that year, economic growth was the first and only priority for both the central and local governments. During the 2000ies, China launched a set of vigorous measures aimed at combating growth-induced pollution and environmental degradation. These reforms intended to address a growing concern of the wealthiest segments of the population, about environmental degradation, and were implemented everywhere, at different speeds. While the 10th Five Year Plan (FYP) starting in 2001 set environmental objectives at the national level, the subsequent 11th plan (2006-2010) moved the incentives for protecting the environment to the local level. It implied that for evaluating the performance of local officers, not only GDP growth rate, fiscal income, industrial value-added, exports, FDI, were considered; They were not any more binding targets with veto-power (\textit{yipiao foujue}). Environmental and energy consumption targets became veto-power targets, which every mayor had to fulfill; Otherwise, the mayor could not pass the end-year evaluation. 


What external circumstances mattered for the realization of these environmental objectives? There is evidence that the cadre rotation system, that switches officials to new positions every three to four years, may be the reason why short term gains, including the extraction of rents from local dirty firms, could be prioritized over long-run ones. Studies have shown that, cities with more educated mayors reach the environmental Kuznets curve turning points at lower levels of per capita income (\cite{Zheng2014-ut}). Last but not least, holding a higher position than local officials, senior managers of state-owned firms perceived themselves as being above the local law. As a result, cities are faced with State-Owned Enterprises (SOEs) that violate environmental regulations, while at the same time gveobs that are essential to local performance. 


This paper provides further evidence, by analyzing the strength of Chinese firms' policy responses to the environmental regulation introduced by the eleventh FYP. More specifically, it shows that, while private firms have been sensitive to the new environmental targets and have made significant progress towards the reduction of their emission of pollutants, SOEs have not taken similar steps. SOEs' inaction in this regard can be viewed as an outcome of the 'Soft Budget Constraint' (SBC), \cite{Kornai2003-nv}. The concept of SBC sees a straightforward application to the objective of sustainable growth: under the condition that they have rational expectations to be refinanced, as in the case of SOEs, enterprises will not be motivated to reach their objective of reducing pollution. The role of Chinese banks in rescuing SOEs and dampening the transition towards a green growth pattern has been documented in \cite{Maurel2019-ap}. 


This paper revisits the issue of the SOEs' behavior and the way these firms react to environmental regulation. We start the analysis by presenting the main characteristics of the Chinese environmental policy before 2006 and up to 2010, with a special emphasis on two key components of this policy. First, the cities targeted by the central government, called 'Two Control Zone' (TCZ herefafter), 175 in number with very poor environmental performance. Second, the SO2 pollution reduction guideline provided in the eleventh FYP by the central government that aligns the motivations of the bureaucrats with the environmental policy. In sections \ref{empirical} and \ref{data}, we discuss the empirical specification and the dataset. Section \ref{analysis} summarizes the primary findings. After the policy shock, local leaders were able overall to decrease the emission of SO2, however in cities with a larger share of SOEs, this outcome did not hold. 


In the subsequent section, thanks to our very rich dataset, we are able to document four different channels, beyond the SBC. Cities, where the presence of SOEs is large, do behave differently because they are predominantly non-TCZ cities. They can be cities where $``$it is hard to kill two birds [$ \ldots $ ] with one stone$"$, in other words, poorer cities. Cities with a large share of SOEs can potentially be characterized by a stronger industrial concentration and/or home firms investing less in greener projects. 


First of all, TCZ cities constitute a special pool of cities that have been selected by the central government for their very poor environmental performance and are subject to a particular vigilance. We expect in these cities a stronger reaction to the environmental regulation. We also pay attention to other policies that can blur the effectiveness of the environmental regulation, noticeably Special Policies Zones (SPZ), and coastal cities. Secondly, wealthier cities are expected to comply more easily with the requirement of a lower emission of pollutants, as wealth raises the demand for cleaner growth. The estimation of Kuznets curves confirms that wealthier cities enjoy SO2 mitigation progress. Third, large firms are in the position to negotiate with local authorities, as they provide work to many employees in the region (\cite{Wang2003-ar}). This bargaining power can translate into weaker compliance with environmental regulation. Fourth, as argued in \cite{Huang1998-pi}, the lack of effective ex-post screening mechanisms in large corporations makes them tend to choose safer innovative projects. In contrast, green projects are usually risky are more likely to be undertaken by small or private firms\footnote{ SOEs are similar to large corporations because they are usually large as well.}. The innovation channel, known also as Porter's hypothesis ( \cite{Porter1995-vr}), is investigated by analyzing the impact of environmental policies on the Total Factor Productivity (TFP). We distinguish the TFP in cities with a large presence of SOEs, a large presence of big corporations, and in TCZ \textit{versus} non-TCZ cities. Finally, the paper draws the main conclusions in section \ref{conclusion}. 


\section{Environmental Policy Background} \label{policy} 
\addcontentsline{toc}{section}{Environmental Policy Background}

\subsection{The TCZ policy under the 10th FYP (2001-2005): a top-down approach}
\addcontentsline{toc}{subsection}{The TCZ policy under the 10th FYP (2001-2005): a top-down approach}

The Chinese policymakers decided to take the environmental issue seriously after the sulfur dioxide (SO2) peak hurt the country in 1995. In no less than three years, the officials in Beijing proposed and ratified a law regulating SO2 emissions. In 1998, the '\textit{Acid Rain Control Zones and Sulfur Dioxide Pollution Control Zones}' policy, abbreviated as '\textit{Two Control Zone}' (TCZ), was implemented by the central government, to control the emissions of this pollutant. While the regulation of SO2 emissions was initially designed to be implemented at the national level, the State Council subsequently chose 175 TCZ cities with very poor environmental records to engage with more effort. Three selection criteria were decided according to the environmental performance preceding the regulations. A city was placed under scrutiny if the average annual ambient SO2 concentration was higher than the national Class II standard (0.06mg/m3), if the daily average ambient SO2 concentration exceeded the national Class III standard (0.25mg/m3)\footnote{ China uses its own air quality standard, which is less stringent than the WHO's standard. China's National Environmental Monitoring Center (CNEMC) has a real-time, hourly air quality data for major cities in China. The real-time data is available at http://www.cnemc.cn/. Major air pollutants, including SO2, NO2, and PM10, are monitored. To evaluate air quality, the Chinese government applies three classes. Class 1 means the yearly SO2 level is less than 0.02 mg/m3, or a daily average of less than 0.05mg/m3. Class 2 is less restrictive. The yearly average should not exceed 0.06 and a daily average of about 0.15. Class 3 corresponds to a bad air quality. The yearly average can exceed 0.10 mg/m3, and the daily average is 0.25. By contrast, the WHO recommends a daily average of less than 0.02mg/m3. For the record, exposure to high SO2 levels dangerously affects health. According to the WHO "SO2 can affect the respiratory system and the functions of the lungs and causes irritation of the eyes. Inflammation of the respiratory tract causes coughing, mucus secretion, aggravation of asthma, and chronic bronchitis and makes people more prone to infections of the respiratory tract".}, or if the city experienced significant SO2 emissions\footnote{ A city was designated as an acid rain control zone if:(1) its average PH value of precipitation was equal or smaller than 4.5;(2) its sulfate deposition was above the critical load(3) its SO2 emissions were large }. 


These 175 cities are primarily concentrated in two areas: northern China, due to the heavy reliance on coal to power the heating system, and southern China, where the urban-industrial centres emit substantial air pollution and are the source of severe acid rains. These TCZ cities cover 1.09 million square kilometres, in 27 provinces, and they account for 11.4$\%$ of the whole of China's territory.


At the national level, the objectives were the following: the emissions of SO2 were expected to decrease successively in 2000 and 2010, and a special role was devoted to TCZ cities, which were granted the responsibility of achieving the national Class II standard of 0.06mg/m3. The quota of SO2 emissions set by the central government in 2000 should not exceed 24.6 million tonnes – compared with 23.7 million tonnes in 1997 – and emissions in 2010 were expected to decrease even more than in 2000. In 2001, the policymakers strengthened the consistency of the environmental policy called the control policy in the 10th FYP (2001-2005). 


TCZ cities were allowed to use four methods to achieve pollution reduction targets. They could shut down polluting plants, install new equipment, use cleaner-burning coal and implement stringent monitoring devices. All power stations with a capacity lower than 50,000 kilowatts, or collieries fueled by coal with a sulfur content of 3$\%$, had to be shut down\footnote{ 338 small power units, 784 product lines in small cement and glass plants, 404 lines in iron and steel plants, and 1422 additional pollution sources had closed and by May 2001, 4492 high-sulfur coal mines had ceased production in the TCZ area (\cite{He2002-ga})}. Furthermore, the central government had the power to cancel construction projects that did not fit the objective of lower SO2 emissions. Industrial plants were forced to meet the environmental standards by installing higher-capacity (more expensive) pollution control equipment\footnote{ The second policy is the installation of flue gas desulfurization (FGD) equipment on new and existing coal-fired power plants. At the end of 2005, FGD equipment had been installed on 46.2 GW of coal-fired electricity generation capacity—12 percent of the total, see \cite{Cao2009-sv}}. Finally, the government carefully monitored the purchase of fuel oil by firms located in TCZ cities. The transportation department was given the charge of supplying fuel oil with a sulfur concentration of less than 2$\%$ or coal with a sulfur concentration of less than 1$\%$. 


Table \ref{table_1} reports the emissions of SO2 during three subsequent FYPs, from 1998 to 2010. The emissions of SO2 rose again after a short drop by -2$\%$ in 1998-2001. By the end of 2000, 102 TCZ cities reached the national Class II standard\footnote{ 84.3$\%$ of the most polluting firms achieved the national target in terms of SO2 emissions (China Environment Yearbook, 2001)}. The entry of China into the WTO in 2001 launched a process of massive industrialisation, economic growth and reduction of poverty, which was at odds with the achievement of the objective of stricter control over pollution. The consequences of the lack of coordination and the focus on economic growth from the local governments led to a historical peak of SO2 emissions in 2005, which rose by a factor of 45$\%$  over 2002-2005.


The poor results of the environmental policy were attributed to the design of the policy itself. Its main flaw was that the objectives set at the national level were not restrictive enough at the local level. As a result economic growth was strongly emphasised by the central government, which did not provide local municipalities with the incentives to enforce not only economic growth, but also control over pollution. Most of the time, those objectives turned out to be contradictory and could not be achieved contemporaneously (\cite{Barbier2019-ce,Brajer2011-wc,Grossman1995-fb,Lee2015-pw}).


\begin{table}[!htb] \centering
\caption{\\ SO2 reduction (\%) during the subsequent FYPs}
\label{table_1}
\begin{adjustbox}{width=\textwidth, totalheight=\textheight-2\baselineskip,keepaspectratio}
\begin{tabular}{lrrrr}
\toprule
             Cities &  1998-2001 &  2002-2005 &  2006-2010 &  Target \\
\midrule
             No TCZ &     21 &     64 &    -11 &  -6 \\
                TCZ &     -7 &     38 &    -15 & -16 \\
 $\text{No Dominated SOE}^ a$ &    -31 &     30 &    -16 & -10 \\
 $\text{Dominated SOE}^ a$ &      6 &     21 &    -17 & -12 \\
        Full Sample &     -2 &     45 &    -13 & -10 \\
        
\bottomrule
\end{tabular}
\end{adjustbox}
\begin{tablenotes} 
 \small 
 \item 
Sources: Author's own computation \\
The list of TCZ is provided by the State Council, 1998.
"Official Reply to the State Council Concerning Acid Rain Control Areas
and Sulfur Dioxide Pollution Control Areas".
The information about the SO2 level are collected using various editions
of the China Environment Statistics Yearbook.
We compute the reduction of SO2 emission using the same methodology
as Chen and al.(2018). \\
$a$ (No) Dominated SOEs cities refer to cities where the 
(output, capital, employment) share of SOEs is (below) above a critical threshold, for instance the 60th decile
 
\end{tablenotes}
\end{table}

\subsection{Decentralization and new incentives to the bureaucrats: the TCZ policy under the 11th FYP (2006-2010)}
\addcontentsline{toc}{subsection}{Decentralization and new incentives to the bureaucrats: the TCZ policy under the 11th FYP (2006-2010)}

In 2006, the central government reconsidered its strategy, changing from a top-down to a bottom-up approach. Echoing the academic literature\footnote{\cite{Dewatripont1999-nq,Alesina2007-xp} are among the first to argue that the features of the mandated tasks largely drive bureaucrats' performance and efforts. The missions must be embedded in a precise interpretation scheme and be linked to performance measures. According to \cite{Alesina2008-eq}, bureaucrats choose their effort level according to two parameters: a concrete objective to reach and the weight of each task in the likelihood to move up the hierarchical leadership ladder.}, which has provided extensive research on the motivations of bureaucrats to implement a policy, the two main differences introduced in the 11th FYP (2006-2010) compared with the previous FYP (2001-2005) are the formulation of a clear pollution reduction guideline for the Chinese provinces and the introduction of an environmental target-based evaluation system for the promotion and career achievement of local officials. This target-based evaluation system aims at promoting the efforts toward the objectives considered to be priorities by the central government. It provides a tool for measuring the success of the local administration, making them accountable. The threat imposed by Beijing forces the mayors and party secretaries to adhere to the national policy. These new incentives are strongly emphasized in \cite{Kahn2015-ok}, who consider that they bear a large responsibilities in the success of the new regulation.


This new focus on environmental concerns from both the central and local governments was followed by immediate and measurable consequences: over 2006 to 2010 the average growth rate of SO2 emissions fell by -13$\%$ (full sample) as reported in table \ref{table_1}, and most TCZ cities (95$\%$ ) were able to reach the national Class II standard SO2 concentration, with no cities reporting values above the national Class III standard (\cite{Ministry_of_Environmental_Protection_of_the_Peoples_Republic_of_China2011-oi}). Local officials in TCZ cities paid more attention to the environmental prejudice of economic growth, they faced a more demanding target: -16$\%$ (TCZ cities) as compared with -6$\%$ (non-TCZ cities) and performed better with respect to the objective, achieving a SO2 reduction of -15$\%$, while non-TCZ cities reached only -11$\%$. 


\section{Empirical Specification} \label{empirical}
\addcontentsline{toc}{section}{Empirical Specification}



Our identification strategy is based upon the qualitative change in the environmental strategy from a top-down to a bottom-up approach in 2006, which split the time span into two sub-periods: 2001-2005 corresponding to the 10th FYP and 2006-2010 corresponding to the 11th FYP. $target_i$ - our treatment variable - is a measure of the intensity of the policy: the reduction mandate from 2006 onwards, available at the provincial level, which therefore allows for geographical differences in treatment intensity. From the information on target available at the province level we proxy the intensity of the regulation at the city level (see below). 


One concern is the influence of the most polluting industries on the probability for a given city to be provided with a more stringent reduction mandate. If such a relationship holds, it results in an endogeneity bias to be addressed: the environmental policy influences the pattern of SO2 emissions, while the share of the most polluting industries determines the other way around the level of pollution and the probability of being required to address this level of pollution in a more stringent way. We add an interaction term, which is the $target_i$ policy times the 11th FYP period times a dummy capturing the most polluting industries in China. 


The spatial variable captures the SO2 emissions for cities faced with different targets. The industrial variable controls for the double causality running from the $target_i$ policy to the emission of pollutant and \textit{vice et versa}. The variable $Period$ measures the effect of the introduction of more stringent and accountable environmental objectives after 2005 and the launch of the 11th FYP. The resulting difference-in-difference (DD) design accounts for these levels of variability and allows us to isolate the effect of stricter environmental policies before and after the 11th FYP: 
\begin{equation} \label{eq:equation_1}
\begin{aligned} 
\text {Log SO2 emission }_{i k t}= & \alpha (Target_{i} \times \text { Polluted sectors }_{k} \times \text { Period }) + \\ 
& \theta {X}_{i k t}+\nu_{ik}+\lambda_{it} +\phi_{kt} +\epsilon_{ikt}
\end{aligned}
\end{equation}

Where $\text {Log SO2 emission }_{i k t}$ is the level of SO2 in city $i$, industry $k$ and at time $t$. The equation includes three right-hand-side variables of interest, a set of control variables, ${X}_{i k t}$, and fixed effects. $\text { Polluted sectors }_{k}$ is a dummy variable taking the value one for heavily polluting industries $k$, and zero for less polluting ones. $\text { Period }$ is a dummy variable, which is set equal to one when $t$ is strictly above 2005. 


We add three control variables usually found in the literature (\cite{Andersen2016-pa,Andersen2017-wf}), which are the $\text{total output}_{ikt}$, $\text{total fixed asset}_{ikt}$, and $\text{employment}_{ikt}$ aggregated at the city $i$, industry $k$ and year $t$. The equation includes a city-year fixed effect $\phi_{it}$, which controls for all city characteristics that differ across cities over time, such as productivity, policies and wages. $\lambda_{kt}$ is an industry-year fixed effect which captures the time-varying industry characteristics, e.g., industry-specific technology and government's industrial policies. With the inclusion of city-industry fixed effects $\nu_{ik}$, we address the time invariant differences between the cities' industries, which are key in our approach: while industrial policies are decided at the central level for the whole country, local municipalities orchestrate their implementation differently from one city to another. In our equation, $\epsilon_{ikt}$ represents the error term. We expect the coefficient $\alpha$ to be negative: cities emit less SO2 after 2005 in more polluting industries. For robustness purposes, we run also specifications with single fixed effects - city, industry and year, less demanding in terms of degrees of freedom, but commonly found in the literature. 


China's political pecking order of firms is enforced through the systematic misallocation of financial resources (\cite{Dollar2007-dr}) with credit allocation being biased in favor of SOEs (\cite{Brandt2003-hu,Ferri2009-lh,Hale2011-ma,Huang2003-oa}), whatever their compliance with the governmental objectives of the FYP. SOEs in China also benefit from more substantial bargaining power when it comes to negotiating the pollution tax system (\cite{Wang2003-ar,Wang2005-yy}). As a result, SOEs are less sensitive to the environmental regulation hardening because of their stronger bargaining power and easier access to bank funding. 


To assess the lower sensitivity of SOEs, besides the spatial, industrial, and time dimensions, we split the sample into two sub-samples, according to the $\text { Share SOE }_{i}$. This term refers to a proxy for the relative share of SOEs in city $i$, above or below a certain threshold\footnote{ We check the robustness of the results by resorting to different thresholds: 60th, 70th, and 80th deciles.}. We expect cities with a larger presence of private firms to react in a more vigorous way, because policymakers put more pressure on them, while SOEs-dominated cities enjoy a softer budget constraint, hence cope more easily with the regulation. The coefficient $\alpha$ should therefore be larger (smaller, or insignificant) in absolute value in the sub-sample where the $\text { Share SOE }_{i}$ is lower (higher). In all regressions, the standard errors are clustered by industry.


\section{Data } \label{data} 
\addcontentsline{toc}{section}{Data }



Our key interest is the SO2 emissions, which are available at the city–industry–year level. Using various data sources, we construct a dataset including environmental, industrial and economic information at the city–industry–year level over the period 2002 to 2007.


\subsection{SO2 emissions}
\addcontentsline{toc}{subsection}{SO2 emissions}

The Ministry of Environmental Protection (MEP) has mandated the State Environmental Protection Agency (SEPA) to collect data on the primary sources of pollutants and waste in China since 1980. The SEPA has monitored firms in 39 major industrial sectors that are considered to be heavy polluters. These firms are required to report basic information, such as company name, address, and output. They also answer a very detailed questionnaire about their emissions of major pollutants (e.g. wastewater, COD, SO2, industrial smoke, and dust). Based on these surveys, the data on pollutants we use are available only at the city level. 


As reported by \cite{Wu2017-bl} and by \cite{Jiang2014-pf}, the resulting dataset embodies 85$\%$ of the emissions of the major pollutants in China. The MEP has implemented strict procedures, such as unexpected visits from experts, to ensure that firms do not misreport their emissions. Having access to the statistics of SO2, a primary air pollutant, our left-hand side variable is $\text {Log SO2 emission }_{i k t}$, which is the logarithm of S02 emissions in city $i$, industry $k$, and year $t$, for 296 four-digits industries, across 228 cities from 2002 to 2007. We define $\text { Polluted sectors }_{k}$ equal to one when an industry emits more than 68,070 tonnes of SO2 (top 25 $\%$ most polluting sectors). 


The emissions of SO2 reached a peak in 2005 at 32.41 million tonnes (China Statistical Yearbook on Environment, 2005). Out of the 522 cities monitored by the Chinese Ministry of Environment, about 400 reported an annual average level of SO2 that met the Grade II national standard (0.06mg/m3) and 33 cities met the worst grade (0.10mg/m3). Two years after the 11th FYP was launched, the situation had slightly changed, according to the Ministry of Environment in its annual report on the state of the environment in China. 79$\%$ of the audited cities met Grade II, which is two percentage points better than in 2005. In regards to the Grade III criteria, less than 1.2$\%$ of the cities were above the threshold, which corresponds to an improvement of four percentage points from 2005. The most polluted cities are located in Shanxi, Guizhou, Inner Mongolia, and Yunnan provinces.

\subsection{Ownerships}
\addcontentsline{toc}{subsection}{Ownerships}

The National Bureau of Statistics of China (NBS) distinguishes manufacturing data with sales above RMB 5 million for non-SOEs and SOEs. The survey contains detailed information about the name, address, four-digit Chinese industrial classification (CIC), ownership, financial variables, output, sales, fixed assets, at the firm level. It is aggregated at the city–industry–year level to be merged with the dataset on SO2 emissions.


\begin{table}[!htb] \centering
\caption{\\ Economic importance of SOE's (in \%)}
\label{table_2}
\begin{adjustbox}{width=\textwidth, totalheight=\textheight-2\baselineskip,keepaspectratio}
\begin{tabular}{lrrr}
\toprule
index & $\text{Output share SOE}_i$ & $\text{Capital share SOE}_i$ & $\text{Employment share SOE}_i$ \\
\midrule
Full sample          &             24.9 &              35.5 &                 29.2 \\
Central              &             28.7 &              41.8 &                 34.2 \\
Northeast            &             26.1 &              38.7  &                33.8 \\
Northwest            &             43.4 &              51.1 &                 49.5 \\
Southwest            &             31.0 &              44.6 &                 36.1 \\
Central              &             28.8 &              41.8 &                 34.2 \\
Eastern              &             15.9 &              24.8 &                 19.1 \\
Western              &             37.7 &              48.4 &                 42.8 \\
No TCZ               &             24.3 &              35.1 &                 29.9 \\
TCZ                  &             25.3 &              35.7 &                 28.7 \\
Concentrated city    &             34.1 &              45.4 &                 37.9 \\
No Concentrated city &             15.1 &              24.9 &                 19.9 \\
\bottomrule
\end{tabular}
\end{adjustbox}
\begin{tablenotes} 
 \small 
 \item  
Sources: Author's own computation \\
The list of TCZ is provided by the State Council, 1998. \\
Output $\text { Share SOE }_{i}$ refers to the ratio of output (respectively capital, employment) of SOEs over the total production (capital, employment) in city $i$.
 \end{tablenotes}
\end{table}



Summary statistics about the economic importance of SOEs is reported in table \ref{table_2}. SOEs represent a big share of the Chinese cities' economy, which varies from 25$\%$ (output) to 35$\%$ (capital). These shares are significantly lower in the richest areas, dropping to 11$\%$ (output) in coastal areas (respectively 18$\%$ capital, 12$\%$ employment), and 15$\%$  in Eastern areas, while they reach 43$\%$ in the Northwest and 37$\%$ in Western areas, which are poor.

Interestingly, the output (resp. capital and employment) share of SOEs in cities, where industrial concentration is high\footnote{The methodology for sampling in two sub-samples, concentrated versus no-concentrated cities, is based upon the computation of an Herfindahl index. Details are exposed in subsection \ref{concentration}}, reaches 34$\%$, as opposed to 15$\%$  in the $``$no-concentrated$"$  cities, and 25$\%$  for the full sample. Also, these SOEs shares are similar in TCZ and non-TCZ cities: about 24-25$\%$  for output, 35$\%$  for capital, and 29-30$\%$  for employment. 


\subsection{Policy variables: TCZ and Target}
\addcontentsline{toc}{subsection}{Policy variables: TCZ and Target}


As documented in the Environmental Policy Background section \ref{policy}, the State Council launched in 1998 a vast policy to curb SO2 emissions and to reduce the acid rain. There were 175 cities called TCZ cities located in 27 provinces designated to provide the subsequent effort for controlling SO2 emissions. Out of the 228 cities in our dataset, 140 are qualified as TCZ cities. Table \ref{appendix} in the appendix provides the list of TCZ cities present in our dataset. $TCZ_{i}$ is a dummy set equal to one if city $i$ belongs to this list. 




\begin{table}[!htb] \centering
    \caption{\\ GDP per capita, population and SO2 emissions}
      \begin{adjustbox}{width=\textwidth, totalheight=\textheight-2\baselineskip,keepaspectratio}
    \label{table_3}
    \begin{tabular}{lrrrrrrrrrrrr}
      \multicolumn{1}{l}{\textbf{\small Panel A:}} \\
      \multicolumn{1}{l}{\textbf{\small GDP per capita and population}} \\
      \toprule
     & \multicolumn{3}{c}{No TCZ} & \multicolumn{3}{c}{TCZ} \\
      & (1)  & (2) & & (3)  & (4) \\
      & 2004-2005 &  2006-2007 & & 2004-2005 &2006-2007  \\
      
      \midrule
      $\text{gdp per capita}_i$    & 16,496 & 22,665 & & 23,992 & 32,984 \\
$\text{population}_i$ &     79 &     83 &  &  158 &    166  \\

      \bottomrule
      \\ %%% Create second table
        \multicolumn{1}{l}{\textbf{Panel B:}} \\
        \multicolumn{1}{l}{\textbf{SO2 emissions (millions of tonnes)}} \\
        \toprule
       & \multicolumn{3}{c}{No TCZ} & \multicolumn{3}{c}{TCZ} \\
      & (1)  & (2) & (2) - (1) & (4)  & (5)  &(5) - (4) \\
      & 2004-2005 &  2006-2007 & &  2004-2005 &2006-2007  \\
\hline \\[-1.8ex] 
Full sample &  2.624  & 2.833  &  0.209 &  9.736 &10.294  &  0.558  \\
Central     &  0.991  & 1.004  &  0.013 & 2.138 & 2.154  &   0.016  \\
Coastal     &  0.556  & 0.632  &  0.076 & 3.763 & 3.860 &   0.097  \\
Northeast   &  0.354  & 0.389  &  0.035 & 0.537 & 0.736 &    0.199  \\
Northwest   &  0.265  &  0.360 &  0.096 & 1.110 & 1.097 &    -0.012  \\
Southwest   &  0.459  & 0.448  &  -0.010 & 2.189 &  2.448 &   0.259  \\
      \bottomrule
      \hline
    \end{tabular}
    \end{adjustbox}
    \begin{tablenotes}
      \small
      \item 
      \footnotesize{
      Sources: Authors' own computation \\
  Panel A: $\text{gdp per capita}_i$ is in RMB and $\text{population}_i$ is in million. \\ $\text{gdp per capita}_i$ and and $\text{population}_i$ are averaged over 2004-2005 and 2006-2007. They are borrowed from the China City Statistical Yearbooks 2002–2007. \\
  Panel B: reported numbers are in millions of tonnes.  \\
  All variables are summed over the years 2004 and 2005 and over the years 2006 and 2007. 
      }
    \end{tablenotes}
\end{table}

Table \ref{table_3} Panel A provides $\text{gdp per capita}_i$ and $\text{population}_i$ in TCZ (non-TCZ cities) and SOEs dominated (SOEs non dominated) cities. SO2 Emissions are reported in Panel B. We notice that TCZ cities are, by definition, making more effort to execute stringent environmental regulations. We get also some preliminary evidence of a positive relationship between wealth and the demand for cleaner environmental goods. In the Chinese case, the evidence is documented in \cite{Hering2014-af}. TCZ cities should therefore emit less SO2\footnote{In section \ref{analysis}, we estimate Kuznets curves to check that richest cities pollute less.}. The same reasoning can be applied to SOEs Dominated cities, which are poorer on average and therefore potentially more polluting than their counterparts. Overall SO2 emissions averaged over 2006-2007 exceed the level in the period before, which suggests the existence of a trend and the need to control for it in the empirical analysis. 


Table \ref{table_3} Panel B gives a more in-depth overview of the SO2 pattern in the major areas of China. Following \cite{Wu2017-bl}, we split the cities between Coastal, Southwest, Central, Northeast, and Northwest areas\footnote{This province breakdown follows the paper of \cite{Wu2017-bl}. The Central provinces are Anhui Henan, Hubei, Hunan, Jiangxi, and Shanxi. The Coastal provinces are Beijing, Fujian, Guangdong, Hainan Hebei, Jiangsu, Shandong, Shanghai, Tianjin, and Zhejiang. The Northeastern provinces are Heilongjiang, Jilin, Liaoning. Northwest are Gansu, Inner Mongolia, Ningxia, Qinghai, Shaanxi, and Xinjiang. The Southwestern parts are Chongqing, Guangxi, Guizhou, Sichuan, Yunnan, and Xizang.}. In our sample, the Coastal area of China is composed of ten provinces and has a total of 68 TCZ cities. This area is the wealthiest part of China: it represents the lion's share of national production and attracts the most significant foreign investment flows. The Southwestern area has five provinces and 24 TCZ cities, while the Central area has six provinces and 38 TCZ cities. The Northern part of China is split into its Western area with six provinces and 13 TCZ cities and Eastern area with three provinces and 11 TCZ cities. 



\subsection{Target-based evaluation system}
\addcontentsline{toc}{subsection}{Target-based evaluation system}



In 2006, the central government provided a clear SO2 pollution reduction guideline for the provinces in China, called the target-based evaluation policy, which the government used to deepen its political ties with the local cadres and to guarantee the fulfillment of the pollution reduction targets. The document stipulated that the provincial leaders had a binding contract with the Ministry of Environment and would bear the responsibility for any failure. Since the guideline, reduction mandates and environmental target-based evaluation system are not available at the city $i$ level, but at the provincial level, we apply the following formula\footnote{Which we borrowed from \cite{Chen2018-bs}}. For $t$ = 2006 or 2007: 
\begin{equation} \label{eq:equation_2}
target_{it} = target_{i} = \Delta SO2_{i, 05 − 10}= \Delta SO2_{p, 05 − 10} \times \sum_{k=1}^{29} \mu_{k} \frac{\text Y_{ki,2005}} {Y_{kp,2005}}
\end{equation}

where $i$ stands for city, $p$ for province and $k$ for the two digit industry ($k$ varies from 1 to 29). The left-hand side of the formula $target_{it}$ evaluates by how much a city should reduce its SO2 emissions between 2005 and 2010, in 10,000 tonnes.   

$\Delta SO2_{p, 05 − 10}$ refers to the reduction mandate at the provincial level. It is available for the 31 provinces of China and over the period 2005-2010 \footnote{ For instance, the province of Shanghai is expected to reduce its SO2 emissions by 13,000 tonnes over the period 2005-2010.}.  $\sum_{k=1}^{29} \mu_{k} \frac{\text Y_{ki,2005}} {Y_{kp,2005}}$ is the share of industrial production $k$, in city $i$, over the total output of industry $k$, in province $p$, multiplied by $\mu_{k}$. $\mu_{k}$ is a weight which reflects each $k$ industry's contribution to the total industrial SO2 emissions. It is set equal to the ratio of SO2 emission in industry $k$ over total SO2 emission. The information about pollution emitted at the two digit level is taken from the MEP dataset. All values refer to 2005.

Table \ref{table_4} provides an overall picture of the efforts required on average by the Chinese cities. Not only TCZ cities, but also cities along the coastal area need to engage in more work to meet the requirement in term of SO2 reduction at the end of the 11th FYP. The majority of cities with a larger share of SOE have the obligation to reduce by a larger extent their emission of SO2.


\begin{table}[!htb] \centering
\caption{\\ Mean Target (1000 thousands tonnes) in SOE dominated \textit{versus} No SOE dominated}
\label{table_4}
\begin{adjustbox}{width=\textwidth, totalheight=\textheight-2\baselineskip,keepaspectratio}
\begin{tabular}{lrrr}
\toprule
{} & All Cities & No SOE dominated & SOE dominated \\
index                &            &                  &               \\
\midrule
Full sample          &   0.120672 &         0.118874 &      0.123933 \\
Central              &   0.089706 &         0.074656 &      0.109946 \\
Coastal              &   0.177910 &         0.180357 &      0.160507 \\
Northeast            &   0.057578 &         0.051720 &      0.068123 \\
Northwest            &   0.056961 &         0.028715 &      0.075238 \\
Southwest            &   0.168342 &         0.118218 &      0.215334 \\
Central              &   0.089706 &         0.074656 &      0.109946 \\
Eastern              &   0.141257 &         0.147305 &      0.115248 \\
Western              &   0.119258 &         0.086123 &      0.142038 \\
No TCZ               &   0.056946 &         0.050289 &      0.071988 \\
TCZ                  &   0.160727 &         0.167522 &      0.149906 \\
Concentrated city    &   0.097203 &         0.061121 &      0.128704 \\
No Concentrated city &   0.145847 &         0.153401 &      0.107238 \\
\bottomrule
\end{tabular}
\end{adjustbox}
\begin{tablenotes} 
 \small 
 \item 
Sources: Author's own computation \\
The list of TCZ is provided by the State Council, 1998. \\
(No) SOE Dominated cities refers to cities where the 
(output, capital, employment) share of SOEs is (below) above a critical threshold, for instance the 60th decile. (No) Concentrated city refers to cities where the Herfindahl index is (below) above a critical threshold, for instance the 60th decile.
\end{tablenotes}
\end{table}
%\end{sidewaystable}

\subsection{Control variables}
\addcontentsline{toc}{subsection}{Control variables}

The literature has listed the key determinants of environmental degradation at the firm level (\cite{Cole2003-ad,Cole2008-pj}). Capital intensity affects both the emissions and intensity of pollution (\cite{Hering2014-af,Andersen2017-wf}). Firms' size matters: large industries emit more pollutants. Besides, we use the NBS industrial classification to sort firms according to the sector they belong to. We rely on the 2002 four-digit CIC and compute total employment, total output, and total net fixed assets, aggregated at the city-industry-year level. The information is generated from the Annual Survey of Industrial Firms (ASIF) conducted by China's NBS for the period from 2002- 2007. 


\section{Empirical Analysis} \label{analysis} 
\addcontentsline{toc}{section}{Empirical Analysis}


\subsection{Main results}
\addcontentsline{toc}{subsection}{Main results}



Table \ref{table_5} (columns 1 to 8) reports the results of equation \ref{eq:equation_1}. The coefficient of interest measures the effect of the environmental target-based policy on the emissions of SO2 in the polluting sectors, with a particular emphasis on cities dominated by SOEs. The triple interaction term $(target_{i} \times \text { Polluted sectors }_{k} \times \text { Period })$ estimate is negative and significant at 5$\%$  (column 1) and 1$\%$  (column 2), meaning that SO2 emissions felt significantly after the launch of the 11th FYP, more in cities where the intensity of the treatment was higher, in line with our expectations. A straightforward calculus shows that the reduction of SO2 reaches about 5\% of the average emission of No Dominated SOEs cities \footnote{5.03\% = 1-exp(0.430*0.118874), with 0.118874 being the average $target_i$ value for No SOEs Dominated cities, see table \ref{table_4}, column 3. Assuming that SOEs dominated cities, where $target_i$ is set equal to 0.1239 millions tonnes, would react the same way to the Target policy, SO2 emission in those cities could also decrease by 5\%.}. Other control variables have the expected signs: economic growth has severely degraded the environment; GDP, employment and fixed assets are correlated with more emissions of SO2. The inclusion of these variables does not affect the main coefficient of interest.




\begin{table}[!htb] \centering
  %\resizebox{1\textwidth}{!}{
    %\begin{threeparttable}
    \caption{\\ 
    Baseline Regression: State versus Private sectors}
      \begin{adjustbox}{width=\textwidth, totalheight=\textheight-2\baselineskip,keepaspectratio}
     \label{table_5}
      \begin{tabular}{@{\extracolsep{5pt}}lcccccccc} 
        \multicolumn{1}{l}{\textbf{Panel A: SOE Dominated sectors}} \\
        \toprule
        & \multicolumn{8}{c}{Dependent variable $\text { SO2 emission }_{i k t}$} \\ 
        \cline{2-9}
            
\\[-1.8ex]
            &\multicolumn{2}{c}{Full sample}&\multicolumn{2}{c}{Output}&\multicolumn{2}{c}{Capital}&\multicolumn{2}{c}{Employment}\\
\\[-1.8ex] & (1) & (2) & (3) & (4) & (5) & (6) & (7) & (8)\\ 
\hline \\[-1.8ex] 
 $output_{cit}$ & 0.144 & $-$0.066 & 0.875$^{*}$ & 0.216 & 1.092$^{**}$ & 0.308 & 0.995$^{**}$ & 0.087 \\ 
  & (0.152) & (0.089) & (0.457) & (0.257) & (0.464) & (0.397) & (0.421) & (0.280) \\ 
  $capital_{cit}$ & 1.481$^{**}$ & 0.971$^{***}$ & $-$3.614$^{***}$ & $-$0.611 & $-$4.106$^{***}$ & $-$0.485 & $-$4.094$^{***}$ & $-$0.393 \\ 
  & (0.730) & (0.366) & (1.088) & (0.617) & (1.146) & (0.773) & (1.104) & (0.708) \\ 
  $labour_{cit}$ & 3.089$^{***}$ & 1.538$^{**}$ & 11.671$^{***}$ & 4.601$^{***}$ & 12.228$^{***}$ & 5.216$^{***}$ & 12.327$^{***}$ & 4.783$^{***}$ \\ 
  & (0.832) & (0.767) & (2.326) & (1.475) & (2.374) & (1.353) & (2.359) & (1.548) \\ 
   $target_c \times \text{Period}$  & $-$0.003 &  & 0.087 &  & 0.012 &  & 0.128 &  \\ 
  & (0.104) &   & (0.381) &   & (0.379) &   & (0.379) &   \\ 
   $target_c \times \text{Polluted}_i$  & 0.438$^{***}$ &  & 0.838$^{**}$ &  & 0.748$^{*}$ &  & 0.839$^{**}$ &  \\ 
  & (0.140) &   & (0.383) &   & (0.390) &   & (0.379) &   \\ 
   $target_c \times \text{Period} \times \text{Polluted}_i$  & $-$0.291$^{**}$ & $-$0.430$^{***}$ & $-$0.430 & 0.009 & $-$0.344 & $-$0.159 & $-$0.468 & $-$0.215 \\ 
  & (0.146) & (0.132) & (0.465) & (0.465) & (0.443) & (0.435) & (0.432) & (0.461) \\ 
 \hline \\[-1.8ex] 
City fixed effects & Yes & No & Yes & No & Yes & No & Yes & No \\ 
Industry fixed effects & Yes & No & Yes & No & Yes & No & Yes & No \\ 
Year fixed effects & Yes & No & Yes & No & Yes & No & Yes & No \\ 
City-year fixed effects & No & Yes & No & Yes & No & Yes & No & Yes \\ 
Industry-year fixed effects & No & Yes & No & Yes & No & Yes & No & Yes \\ 
City-industry fixed effects & No & Yes & No & Yes & No & Yes & No & Yes \\ 
Observations & 30,676 & 30,676 & 9,165 & 9,165 & 9,149 & 9,149 & 9,011 & 9,011 \\ 
R$^{2}$ & 0.346 & 0.851 & 0.377 & 0.872 & 0.376 & 0.868 & 0.373 & 0.869 \\ 
        \bottomrule
        \\ %%% Create second table
        \multicolumn{1}{l}{\textbf{Panel B: Private Dominated sectors}} \\
        \toprule
        & \multicolumn{8}{c}{Dependent variable $\text { SO2 emission }_{i k t}$} \\ 
        \cline{2-9}
            
\\[-1.8ex]
            &\multicolumn{2}{c}{Full sample}&\multicolumn{2}{c}{Output}&\multicolumn{2}{c}{Capital}&\multicolumn{2}{c}{Employment}\\
\\[-1.8ex] & (1) & (2) & (3) & (4) & (5) & (6) & (7) & (8)\\ 
\hline \\[-1.8ex] 
  $output_{cit}$ & 0.144 & $-$0.066 & 0.116 & $-$0.077 & 0.107 & $-$0.077 & 0.113 & $-$0.073 \\ 
  & (0.152) & (0.089) & (0.134) & (0.094) & (0.130) & (0.090) & (0.130) & (0.095) \\ 
  $capital_{cit}$ & 1.481$^{**}$ & 0.971$^{***}$ & 1.885$^{***}$ & 0.942$^{**}$ & 1.930$^{***}$ & 0.905$^{**}$ & 1.744$^{***}$ & 0.911$^{**}$ \\ 
  & (0.730) & (0.366) & (0.575) & (0.445) & (0.557) & (0.401) & (0.551) & (0.441) \\ 
  $labour_{cit}$ & 3.089$^{***}$ & 1.538$^{**}$ & 2.592$^{***}$ & 1.423$^{*}$ & 2.567$^{***}$ & 1.329$^{*}$ & 2.611$^{***}$ & 1.376 \\ 
  & (0.832) & (0.767) & (0.718) & (0.798) & (0.707) & (0.778) & (0.731) & (0.847) \\ 
   $target_c \times \text{Period}$  & $-$0.003 &  & $-$0.050 &  & $-$0.031 &  & $-$0.049 &  \\ 
  & (0.104) &   & (0.105) &   & (0.105) &   & (0.105) &   \\ 
   $target_c \times \text{Polluted}_i$  & 0.438$^{***}$ &  & 0.388$^{***}$ &  & 0.394$^{***}$ &  & 0.369$^{**}$ &  \\ 
  & (0.140) &   & (0.148) &   & (0.145) &   & (0.150) &   \\ 
   $target_c \times \text{Period} \times \text{Polluted}_i$  & $-$0.291$^{**}$ & $-$0.430$^{***}$ & $-$0.250$^{*}$ & $-$0.429$^{***}$ & $-$0.284$^{*}$ & $-$0.430$^{***}$ & $-$0.256$^{*}$ & $-$0.434$^{***}$ \\ 
  & (0.146) & (0.132) & (0.150) & (0.141) & (0.151) & (0.140) & (0.151) & (0.144) \\ 
 \hline \\[-1.8ex] 
City fixed effects & Yes & No & Yes & No & Yes & No & Yes & No \\ 
Industry fixed effects & Yes & No & Yes & No & Yes & No & Yes & No \\ 
Year fixed effects & Yes & No & Yes & No & Yes & No & Yes & No \\ 
City-year fixed effects & No & Yes & No & Yes & No & Yes & No & Yes \\ 
Industry-year fixed effects & No & Yes & No & Yes & No & Yes & No & Yes \\ 
City-industry fixed effects & No & Yes & No & Yes & No & Yes & No & Yes \\ 
Observations & 30,676 & 30,676 & 21,511 & 21,511 & 21,527 & 21,527 & 21,665 & 21,665 \\ 
R$^{2}$ & 0.346 & 0.851 & 0.355 & 0.855 & 0.359 & 0.856 & 0.358 & 0.856 \\ 
    \end{tabular}
    \end{adjustbox}
    \begin{tablenotes}
      \small
      \item 
      Note: $^{*}$ Significance at the 10\%, $^{**}$ Significance at the 5\%, $^{***}$ Significance at the 1\% Heteroskedasticity-robust standard errors in parentheses are clustered by city
\end{tablenotes}
\end{table}




Our key assumption is that the effectiveness of the policy is lower in cities dominated by SOEs, which face a softer budget constraint. We expect, therefore, a lesser coefficient (smaller in absolute value, or insignificant) for those cities. To test this assumption, we compute for each city the SOEs' output share, share of capital, and share of employment. Then we split the sample into two sub-samples; SOEs dominated sub-sample (table \ref{table_5}, Panel A) is made of cities where the SOEs' output share (respectively capital and employment) is above the 60th decile\footnote{ Similar results hold for the 70th and 80th deciles, they are available upon request } of the total distribution, while non-SOEs sub-sample (table \ref{table_5}, Panel B) includes the remaining cities, below the 60th decile. 


Columns 3 and 4 report the estimates obtained when the 60th decile is based upon SOEs' industrial output share (resp. SOEs' capital share in columns 5-6 and SOEs' employment share in columns 7-8). The coefficient of interest remains negative and significant in the sub-sample of cities with a smaller presence, below the 60th decile. It becomes insignificant in the sub-sample of cities with a stronger presence of SOEs, above the 60th decile, meaning that the policy is attenuated in the polluting sectors where the presence of SOEs is large. This findings confirm that SOEs can adopt business strategies less constrained by the new regulation than private firms as they enjoy a preferential treatment. They do not need to reduce, cut or relocate the production because they receive financial support from the local government. Softer credit constraint helps them to absorb the costs linked to the policy more efficiently. 

\subsection{Testing for parallel trends}
\addcontentsline{toc}{subsection}{Testing for parallel trends}

We must check that our strategy satisfies the parallel trends assumption, by showing that SO2 emissions trajectories are not deviated before the treatment (i.e., before the introduction of local environmental regulations). One might think for instance, that certain local governments anticipated the implementation of the environmental regulation and decided to enforce it before the treatment year. The test for the parallel trend assumption consists of replacing the treatment variable $\text { Period }$ with yearly dummies. The new specification becomes:

\begin{equation} \label{eq:equation_3}
\begin{aligned}
 \text {Log SO2 emission }_{i k t}= & \sum_{t=2002}^{2007} \alpha (Target_{i}  \times \text { Polluted sectors }_{k} \times year _{t}) + \\
 & \theta {X}_{i k t}+\nu_{ik}+\lambda_{it} +\phi_{kt} +\epsilon_{ikt}
 \end{aligned}
\end{equation}

In which $year_t$ is a dummy set equal to one when $t$ ranging from years 2003 to 2007. 


The estimate of $\alpha$ captures the log of SO2 emissions in the whole sample and in the subsamples of SOEs and non-SOEs cities before the policy was implemented. If the parallel trend assumption holds, the coefficient $\alpha$ should not be significant before 2006. Table \ref{table_6} reports the results. The coefficients are all insignificant at the usual levels before the treatment year, validating the parallel trend assumption. In the remaining columns, we split our sample between SOEs cities (column 3, 5 and 7) and non-SOEs cities (column 2, 4, 6). In the even columns, the coefficients for non-SOEs cities are negative, and significant from 2006 onwards, suggesting the early effect of the policy immediately after its introduction. By contrast, the policy does not affect the cities where the presence of SOEs is large (odd columns). Estimates are obtained from specifications that control for output, fixed assets and employment.


\begin{table}[!htb] \centering
  \caption{\\ Test of parallel trend assumption} 
\label{table_6}
\begin{adjustbox}{width=\textwidth, totalheight=\textheight-2\baselineskip,keepaspectratio}
\begin{tabular}{@{\extracolsep{5pt}}lccccccc} 
\\[-1.8ex]\hline 
\hline \\[-1.8ex] 
 & \multicolumn{7}{c}{Estimate of $\alpha$, equation \ref{eq:equation_3} on page \pageref{eq:equation_3}} \\ 
\cline{2-8}
            
\\[-1.8ex]
            &\multicolumn{1}{c}{}&\multicolumn{2}{c}{Output}&\multicolumn{2}{c}{Capital}&\multicolumn{2}{c}{employment}\\
\\[-1.8ex] & (1) & (2) & (3) & (4) & (5) & (6) & (7)\\
 \\[-1.8ex]$year_t$ varying from: & Full sample & No SOE & SOE & No SOE & SOE & No SOE & SOE\\
 \hline \\[-1.8ex] 
  2003 & $-$0.230 & $-$0.232 & $-$0.581 & $-$0.214 & $-$0.326 & $-$0.250 & $-$0.308 \\ 
  & (0.222) & (0.248) & (0.838) & (0.243) & (0.833) & (0.242) & (0.846) \\ 
  2004 & $-$0.156 & $-$0.174 & 0.148 & $-$0.197 & 0.498 & $-$0.229 & 0.498 \\ 
  & (0.222) & (0.240) & (0.808) & (0.237) & (0.772) & (0.239) & (0.807) \\ 
  2005 & $-$0.347 & $-$0.364 & $-$0.290 & $-$0.350 & $-$0.137 & $-$0.307 & $-$0.336 \\ 
  & (0.258) & (0.277) & (0.886) & (0.273) & (0.884) & (0.271) & (0.905) \\ 
  2006 & $-$0.598$^{**}$ & $-$0.645$^{**}$ & $-$0.082 & $-$0.625$^{**}$ & $-$0.019 & $-$0.630$^{**}$ & $-$0.246 \\ 
  & (0.263) & (0.279) & (0.950) & (0.276) & (0.956) & (0.276) & (0.978) \\ 
  2007 & $-$0.715$^{***}$ & $-$0.686$^{**}$ & $-$0.308 & $-$0.703$^{**}$ & $-$0.247 & $-$0.712$^{**}$ & $-$0.273 \\ 
  & (0.265) & (0.280) & (0.926) & (0.276) & (0.902) & (0.282) & (0.947) \\ 
 \hline \\[-1.8ex] 
City-year fixed effects & Yes & Yes & Yes & Yes & Yes & Yes & Yes \\ 
Industry-year fixed effects & Yes & Yes & Yes & Yes & Yes & Yes & Yes \\ 
City-industry fixed effects & Yes & Yes & Yes & Yes & Yes & Yes & Yes \\ 
Observations & 30,676 & 21,511 & 9,165 & 21,527 & 9,149 & 21,665 & 9,011 \\ 
R$^{2}$ & 0.852 & 0.855 & 0.872 & 0.856 & 0.868 & 0.856 & 0.869 \\ 
\hline 
\hline \\[-1.8ex] 
\end{tabular}
\end{adjustbox}
\begin{tablenotes} 
 \small 
 \item 
\footnotesize{
Due to limited space, only the coefficients of interest are presented. $^{*}$ Significance at the 10\%, $^{**}$ Significance at the 5\%, $^{***}$ Significance at the 1\%. Heteroscedasticity-robust standard errors in parentheses are clustered by city 
}
 
\end{tablenotes}
\end{table}

\subsection{Diffusion channels at work under the Soft Budget Constraint} \label{diffusion}
\addcontentsline{toc}{subsection}{Diffusion channels at work under the Soft Budget Constraint}




Four diffusion channels are documented in this section. First, we look at TCZ cities, where we expect a stronger reaction to environmental regulation. We pay attention to other policies run by the central government such as Special Policies Zones (SPZ) and coastal cities, where the emphasis is put on the objective of growth. Second, we estimate Kuznets curves and validate the assumption that wealthier cities enjoy SO2 mitigation progress. Third, large firms are in a better position to negotiate with local authorities. Following this statement, we check whether the bargaining power translates into a weaker compliance with the environmental regulation. Fourth, we verify the Porter’s theory, according to which environmental regulation can be accommodated for through investment in green projects, which enhance productivity. The latter are more likely to be taken by private firms, as large corporations like SOEs tend to choose safer innovative projects. Beyond the SBC, we investigate to what extent SOEs response to environmental regulation is channeled through these four mechanisms. All reported results are obtained with city-year, industry-year, and city-industry fixed effects\footnote{Results with city, industry, and year fixed effects are available upon request}. 


\subsubsection{First mechanism: policies run in the cities}
\addcontentsline{toc}{subsubsection}{First mechanism: policies run in the cities}



Table \ref{table_7} provides evidence that our results are sensitive to the political incentives that cities are facing. We consider three categories of cities: TCZ cities, Special Policies Zone (SPZ) cities, and finally cities close to the sea. While pollution reduction is clearly set as the top political priority for TCZ cities, it can compete for SPZ cities and cities far away from the sea with other political aims. The latter include the 'Go West' policy, which refers to a strategy launched in 2000, when the Chinese government decided to boost the economy of the western areas by pouring billions of US dollars into infrastructure, roads, facilities, and improving the skills of the workers (\cite{Chen2018-ki}). This strategy provides incentives to firms, and more particularly, SOEs to downsize the production in favour of the new (attractive) cities located in China's Western hinterlands. We consider $Coastal_i$ which is one if city $i$ is away from the hinterland and close to the sea, which has historically always been a very attractive area.


The other policy is called SPZ. It aims at boosting the attractiveness of SPZ cities for foreign firms, exporters, or high tech firms, which benefit from lower taxes, access to cheaper credits, or subsidies, amongst others (\cite{Wang2008-gv,Hering2014-af})\footnote{ Those SPZ include High-technology Industry Development Areas, Economic and Technological Development Areas, and Export Processing Zones.}. We define $SPZ_i$ equal to one if city $i$ belongs to the SPZ. In our sample we have 60 SPZ cities, and 35 cities adjacent to the sea.
 

%\begin{sidewaystable}%[!htb] \centering
\begin{table}[!htb] \centering 
  \caption{\\ Effects of the policies: TCZ, Go West, and SPZ } 
\label{table_7}
\begin{adjustbox}{width=\textwidth, totalheight=\textheight-2\baselineskip,keepaspectratio}
\begin{tabular}{@{\extracolsep{5pt}}lcccccc} 
\\[-1.8ex]\hline 
\hline \\[-1.8ex] 
 & \multicolumn{6}{c}{Dependent variable $\text { SO2 emission }_{i k t}$} \\ 
\cline{2-7}
            
\\[-1.8ex]
            &\multicolumn{1}{c}{TCZ}&\multicolumn{1}{c}{No TCZ}&\multicolumn{1}{c}{ Coastal }&\multicolumn{1}{c}{No  Coastal}&\multicolumn{1}{c}{ SPZ }&\multicolumn{1}{c}{No  SPZ}\\
\\[-1.8ex] &  (1) &  (2) &  (3) &  (4) &  (5) &  (6)\\ 
\hline \\[-1.8ex] 
   $target_c \times \text{Period} \times \text{Polluted}_i$  &  $-$0.413$^{***}$ &   $-$1.415 &  $-$0.356$^{**}$ &   $-$0.714$^{**}$ &   $-$0.412$^{***}$ & $-$0.844$^{***}$ \\ 
  &   (0.138) &   (1.059) &  (0.157) &   (0.278) &   (0.157) &   (0.324) \\ 
 \hline \\[-1.8ex] 
City-year fixed effects &  Yes &  Yes &  Yes &  Yes &  Yes &  Yes \\ 
Industry-year fixed effects &  Yes &   Yes &   Yes &  Yes &   Yes &   Yes \\ 
City-industry fixed effects &   Yes &   Yes &   Yes &   Yes &   Yes &   Yes \\ 
Observations &   23,333 &   7,343 &   17,915 &   12,761 &   14,513 &   16,163 \\ 
R$^{2}$ &  0.847 &   0.892 &   0.852 &  0.870 &  0.846 &   0.871
 \\ 
\hline 
\hline \\[-1.8ex] 
\end{tabular}
\end{adjustbox}
\begin{tablenotes} 
 \small 
 \item
\footnotesize{
$^{*}$ Significance at the 10\%, $^{**}$ Significance at the 5\%, $^{***}$ Significance at the 1\%. Heteroscedasticity-robust standard errors in parentheses are clustered by city 
}
 
\end{tablenotes}
\end{table}
%\end{sidewaystable}



Our coefficient of interest remains negative and significant at 1$\%$ for TCZ cities (column 1). It is no more significant for no TCZ cities (column 2). This result is important as it suggests that Target and TCZ policies (local and central level) are complementary.  Interestingly, the reaction of cities located in the hinterlands (No Coastal) to the regulation stands at -0.714, to be compared to -0.356 for Coastal areas. Finally, as expected, the trade-off between growth and pollution reduction is biased towards growth in SPZ cities, where the coefficient of interest is lower, in absolute value, than in non SPZ cities (-0412 for SPZ versus -0.888 for No SPZ). These findings suggest the possible influence of the policies on the performance of the cities dominated by SOEs\footnote{The output share of SOEs in TCZ is 25.3\%, 31\% in the hinterlands, 26\% in SPZs, whereas the average output share of SOEs in the total sample reaches 25\%.}.


\subsubsection{Second mechanism: level of concentration} \label{concentration}
\addcontentsline{toc}{subsubsection}{Second mechanism: level of concentration}





In this section, we look at the effect of size and industrial concentration on environmental regulation effectiveness. Large corporations are more likely to select safer innovative projects, while green and risky projects are undertaken by small or private firms. Besides, large firms can influence local authorities concerning the effective enforcement of environmental regulation, which is also an expression of the SBC. If this is true, private firms should adjust more to the regulation than SOEs. 


Our indicator of industrial concentration is based upon an Herfindahl index, computed as the average of the sum of the squared market share of industry $k$ in city $i$ over the period 2002-2005. Its values are ranging in an interval of 0.016 to 0.82. We define (low) concentrated cities as referring to cities above (below) the 60th decile of this indicator. We re-run our equation of interest in the sub-samples of concentrated versus no concentrated cities. Results are reported in table \ref{table_8}. 




\begin{table}[!htb] \centering
  \caption{\\ Industrial concentration and environmental regulation effectiveness} 
\label{table_8}
\begin{adjustbox}{width=\textwidth, totalheight=\textheight-2\baselineskip,keepaspectratio}
\begin{tabular}{@{\extracolsep{5pt}}lcccc} 
\\[-1.8ex]\hline 
\hline \\[-1.8ex] 
 & \multicolumn{2}{c}{Dependent variable $\text { SO2 emission }_{i k t}$} \\ 
\cline{2-3}
            
\\[-1.8ex]
            &\multicolumn{1}{c}{Concentrated}&\multicolumn{1}{c}{No Concentrated}\\
\\[-1.8ex] & (1) & (2)\\ 
\hline \\[-1.8ex] 
   $target_c \times \text{Period} \times \text{Polluted}_i$  & 0.208 & $-$0.504$^{***}$ \\ 
  & (0.396) & (0.143) \\ 
 \hline \\[-1.8ex] 
City-year fixed effects &  Yes &  Yes \\ 
Industry-year fixed effects &  Yes  & Yes \\ 
City-industry fixed effects &  Yes  & Yes \\ 
Observations  & 9,181 & 21,495 \\ 
R$^{2}$  & 0.888 & 0.846 \\ 
\hline 
\hline \\[-1.8ex] 
\end{tabular}
\end{adjustbox}
\begin{tablenotes} 
 \small 
 \item
\footnotesize{$^{*}$ Significance at the 10\%, $^{**}$ Significance at the 5\%, $^{***}$ Significance at the 1\%. Heteroscedasticity-robust standard errors in parentheses are clustered by city 
}
 
\end{tablenotes}
\end{table}



Results confirm that size and industrial concentration matter. In cities characterized by a low degree of industrial concentration, i.e. below the 60th decile of the Herfindahl index\footnote{ Similar results hold for the 70th and 80th deciles, they are available upon request }, polluted sectors are sensitive to environmental regulation. The coefficient of interest reaches -0.504 (table \ref{table_8}, column 2), it is highly significant at 1$\%$. In concentrated cities it falls to 0.208 (column 1) not significantly different from zero, suggesting that larger companies are in a stronger bargaining position and can impose their own objectives. But SOEs are large companies: their output share in concentrated cities reaches 34\%, while it is set equal to only 15\% in no concentrated cities. 

\subsubsection{Third mechanism: Kuznets curve}
\addcontentsline{toc}{subsubsection}{Third mechanism: Kuznets curve}



We address the concern that cities dominated by SOEs, which are poorer on average\footnote{Our dataset documents that the GDP per capita of cities dominated by SOEs is on average lower by RMB7 000, being RMB31 700 for No SOE dominated cites and RMB24 000 for SOE dominated cities}, could be less sensitive to the environmental regulation because of the correlation between the wealth of the inhabitants and the sensitivity to the environment. Empirical evidence about this correlation is extensive and widely reported in the Kuznets curve literature: wealthier households have the financial capacity to consume in line with their preferences for goods and services that protect the environment (\cite{Berger2019-jl,Chen2018-ki}), or they can escape from polluted cities (\cite{Chen2017-ro}).


We use equation \ref{eq:equation_4} below to estimate the relationship between a Chinese city's emission of SO2 and its characteristics, including log per capita income and squared log per capita income. Following the academic literature studying the environmental Kuznets curve, this allows us to test for whether there is a $``$turning point$"$ such that when a city's per capita income exceeds this turning point level, the association between economic growth and pollution becomes negative. In table \ref{table_9} reported below, we find that wealthier cities enjoy SO2 mitigation progress and that the key turning point ranges from US$\$$ 3481 to US$\$$ 5625\footnote{In \cite{Kahn2016-fi} the turning point is found to be equal to US$\$$ 10 000. It is computed from data on particulate matter annual mean concentration (PM10).}. 

\begin{equation} \label{eq:equation_4}
\begin{aligned}
\text {Log SO2 emission }_{i k t} = & \alpha \text{(ln gdp per cap)}_{ct}  + \beta\text{(ln gdp per cap)}^2_{ct}  + \gamma \text{(ln population)}_{ct} \\ 
& + \nu_{c}+\lambda_{k}+\phi_{t}+\epsilon_{i k t}
\end{aligned}
\end{equation}

Moreover, further analysis in the table \ref{table_9} documents that cities' characteristics – TCZ versus non-TCZ, concentrated versus no concentrated, and SOEs (private firms) dominated cities – matter for the existence of the environmental Kuznets curve. For non TCZ, cities with a high level of industrial concentration, and more importantly high (be it based upon output, capital or employment) share of SOEs, we are not able to detect a turning point, e.g. a level of per capita income, above which the relationship between local economic growth and pollution levels reverses and becomes negative. \footnote{We notice also that the share of SOEs dominated cities below the turning point is higher than its counterpart (below it), whatever the methdology for computing it.}  


\begin{sidewaystable}%[!htb] \centering
  \caption{\\ Kuznets curves} 
\label{table_9}
\begin{adjustbox}{width=\textwidth, totalheight=\textheight-2\baselineskip,keepaspectratio}
\begin{tabular}{@{\extracolsep{5pt}}lcccccccccc} 
\\[-1.8ex]\hline 
\hline \\[-1.8ex] 
 & \multicolumn{10}{c}{Dependent variable $\text { SO2 emission }_{i k t}$} \\ 
\cline{2-11}
            
\\[-1.8ex]
            &\multicolumn{2}{c}{City}&\multicolumn{2}{c}{Concentration}&\multicolumn{2}{c}{Output}&\multicolumn{2}{c}{Capital}&\multicolumn{2}{c}{Employment}\\
\\[-1.8ex] & (1) & (2) & (3) & (4) & (5) & (6) & (7) & (8) & (9) & (10)\\
 \\[-1.8ex]& TCZ & No TCZ & Concentrated & No Concentrated & SOE dominated & SOE No dominated & SOE dominated & SOE No dominated & SOE dominated & SOE No dominated\\
 \hline \\[-1.8ex] 
  $\text{(ln gdp per cap)}_{ct}$  & 2.708$^{***}$ & 0.269 & $-$0.626 & 3.391$^{***}$ & $-$0.096 & 2.611$^{***}$ & 1.437 & 2.559$^{***}$ & 0.434 & 2.984$^{***}$ \\ 
  & (0.927) & (1.428) & (1.192) & (0.971) & (1.784) & (0.837) & (1.882) & (0.847) & (1.555) & (0.854) \\ 
   $\text{(ln gdp per cap)}^2_{ct}$  & $-$0.132$^{***}$ & 0.002 & 0.039 & $-$0.158$^{***}$ & 0.017 & $-$0.127$^{***}$ & $-$0.062 & $-$0.126$^{***}$ & $-$0.015 & $-$0.143$^{***}$ \\ 
  & (0.045) & (0.071) & (0.061) & (0.047) & (0.090) & (0.041) & (0.096) & (0.041) & (0.080) & (0.042) \\ 
   $\text{(ln population)}_{ct}$  & 0.062 & 0.011 & 0.328$^{*}$ & 0.026 & 0.043 & 0.040 & $-$0.093 & 0.055 & 0.054 & 0.036 \\ 
  & (0.103) & (0.168) & (0.191) & (0.102) & (0.211) & (0.096) & (0.221) & (0.094) & (0.218) & (0.092) \\ 
 \hline \\[-1.8ex] 
turning point RMB & 28795 & - & - & 45396 & - & 30264 & - & 24867 & - & 35190 \\ 
turning point Dollar & 3568 & - & - & 5625 & - & 3750 & - & 3081 & - & 4361 \\ 
SOEs above (below) & (43\%, 57\%) & - & - & (11\%, 89\%)  & - & (41\%, 59\%)  & - & (49\%, 51\%) & - & (27\%, 73\%)  \\ 
%Cities above (below)  & (38\%, 62\%) & - & - & (17\%, 83\%) & - & (37\%,63\%) & - &  (44\%, 56\%) & - & (26\%, 74\%) \\ %
City fixed effects & Yes & Yes & Yes & Yes & Yes & Yes & Yes & Yes & Yes & Yes \\ 
Industry fixed effects & Yes & Yes & Yes & Yes & Yes & Yes & Yes & Yes & Yes & Yes \\ 
Year fixed effects & Yes & Yes & Yes & Yes & Yes & Yes & Yes & Yes & Yes & Yes \\ 
Observations & 22,865 & 7,330 & 9,153 & 21,042 & 9,160 & 21,035 & 9,137 & 21,058 & 8,993 & 21,202 \\ 
R$^{2}$ & 0.344 & 0.415 & 0.405 & 0.340 & 0.376 & 0.353 & 0.375 & 0.357 & 0.372 & 0.357 \\ 
\hline 
\hline \\[-1.8ex] 
\end{tabular}
\end{adjustbox}
\begin{tablenotes} 
 \small 
 \item
\footnotesize{
\textit{SOEs above (below)} indicates the share of SOEs dominated cities above and below the turning points, based on the \textit{GDP per capita} of 2007. 
%\textit{Cities above (below)} refers to the share of cities above and below the turning points, whatever the ownerships status, based on the \textit{GDP per capita} of 2007.%
Due to limited space, only the coefficients of interest are presented $^{*}$ Significance at the 10\%, $^{**}$ Significance at the 5\%, $^{***}$ Significance at the 1\%. Heteroscedasticity-robust standard errors in parentheses are clustered by city 
}
 
\end{tablenotes}
\end{sidewaystable}

\subsubsection{Fourth mechanism: environmental regulation-induced TFP improvement}
\addcontentsline{toc}{subsubsection}{Fourth mechanism: environmental regulation-induced TFP improvement}



The evidence about the correlation between pollution abatement on one hand and productivity (scale economy and innovation) on the other hand is considerable, with potentially, a positive or negative sign. For a positive association, the rationale is the following: innovation aims at producing at a lower cost, allowing companies to use fewer inputs, and less dirty energy per unit of output. By imposing a lower strict limit for the emission of pollutants, the new regulation forces the firms to upgrade or leave the market (\cite{Andersen2016-pa,Andersen2017-wf,Cole2008-pj}). This theory is also known as the Porter hypothesis (\cite{Porter1995-vr}). However, the correlation can also be negative. According to the compliance cost theory, if the cost of environmental regulation impedes the improvement of productivity, it results in a decline in industrial performance. A recent paper, \cite{Yang2020-uw} show that the carbon emission trading system launched in 2017 verifies the Porter hypothesis in that it leads to an expansion of the employment scale and reduces the carbon emissions. 


However, these mechanisms may work only for private firms. There is a large body of literature showing that Chinese SOEs report lower economic performances (\cite{Zhang2004-ij,Dougherty2007-qu,Qian1996-ab}) and lower TFP. Indeed the objective function does not focused on profit maximisation, and the soft budget constraint implies that other emphases are put on competing objectives such as employment, social protection and incumbent protection, leaving aside productivity improvement. 

To disentangle these different assumptions, we estimate the following equation \ref{eq:equation_5}:

\begin{equation} \label{eq:equation_5}
TFP_{fikt}=\alpha\left(\text {Target}_{i} \times \text {Polluted sectors}_{k} \times \text { Period }\right)+ \zeta_{f}+ \nu_{i k}+\lambda_{i t}+\phi_{k t}+\epsilon_{i k t}
\end{equation}

Where the dependent variable $TFP_{fikt }$ is the firm, $f$, productivity level computed with the Olley–Pakes algorithm (\cite{Olley1996-yl}) at the firm-city-industry-time level. The panel structure of our dataset allows us to address the endogeneity issues. First, the inclusion of city-time ($\lambda_{i t}$) fixed effects is particularly important because firms in a city faced with stronger regulatory requirement are more likely to be located in industrial areas prone to factors associated with citywide emission trends. Second, the inclusion of industry-time ($\phi_{k t}$) and city-industry ($\nu_{i k}$) fixed effects remove the trends among all firms in a particular industry that are unrelated to the environmental policy. Finally the inclusion of firms' fixed effects ($\zeta_{f}$) remove all unobserved factors contributing to a firm's TFP within a city and these effects are allowed to vary over time\footnote{Another potential bias results from the fact that firms more severely affected by the regulation may exit the market. In this case, the regression will impute an increased productivity to the regulation. Firm-level observations in our dataset allow us to probe this eventuality by investigating various samples of firms that operated throughout the entire period (stayers) and that did not operate in either the first year (entrants) or the last year (dropouts). Our results are robust and available upon request.}. 


Table \ref{table_10} reports the main coefficients of interest from equation \ref{eq:equation_5}, using firm level data over the period 2002-2007. Positive values of the coefficients imply that the target-based regulation led to an increase in the TFP variable, validating the Porter hypothesis according to which strict environmental regulation facilitates technological innovation and compensate for the cost of environmental protection, whereas negative values indicate that the cost of environmental protection faced by enterprises is harmful to investment in innovation and productivity improvement. 


Previous discussion has shown that the effect of the policy on SO2 emission is not homogeneous across cities depending on the status (TCZ versus non-TCZ), level of development (coastal and SPZ cities, cities below and above Kuznets turning points). Therefore, we control for this heterogeneity by distinguishing different sub-samples. In table \ref{table_10}, we compute the effects of the target-based policy on TFP for SOEs versus private firms. Panel A gathers firms belonging to TCZ cities (versus non-TCZ cities), and Panel B coastal (versus no-coastal). Panel C refers to firms belonging to cities where the level of industrial concentration is high as opposed to low. Finally, we run our model using different turning points from table \ref{table_9}, and the results are presented in a separate table \ref{table_11}. 




\begin{table}[!htb] \centering
  %\resizebox{1\textwidth}{!}{
    %\begin{threeparttable}
    \caption{\\ Reduction mandate and TFP}
      \begin{adjustbox}{width=\textwidth, totalheight=\textheight-2\baselineskip,keepaspectratio}
     \label{table_10}
      \begin{tabular}{@{\extracolsep{5pt}}lcccc}  
        \multicolumn{1}{l}{\textbf{Panel A: TCZ versus non-TCZ}} \\
        \toprule
        & \multicolumn{4}{c}{Dependent variable $\text { TFP }_{fikt}$} \\ 
\cline{2-5}
            
\\[-1.8ex]
            &\multicolumn{2}{c}{SOE}&\multicolumn{2}{c}{PRIVATE}\\
\\[-1.8ex] & (1) & (2) & (3) & (4)\\
 \\[-1.8ex]& TCZ & NO TCZ & TCZ & NO TCZ\\
 \hline \\[-1.8ex] 
   $target_c \times \text{Period} \times \text{Polluted}_i$  & 0.144$^{***}$ & $-$0.419 & $-$0.022 & $-$0.421$^{**}$ \\ 
  & (0.050) & (0.429) & (0.021) & (0.188) \\ 
 \hline \\[-1.8ex] 
Firm & Yes & Yes & Yes & Yes \\ 
City-industry &Yes & Yes & Yes & Yes \\ 
City-time & Yes & Yes & Yes & Yes \\ 
time-industry & Yes & Yes & Yes & Yes \\ 
Observations & 32,078 & 9,410 & 517,652 & 89,657 \\ 
R$^{2}$ & 0.953 & 0.961 & 0.861 & 0.869 \\

        \bottomrule
        \\ %%% Create second table
        \multicolumn{1}{l}{\textbf{Panel B: Coastal  versus non - Coastal}} \\
        \toprule
        & \multicolumn{4}{c}{Dependent variable $\text { TFP }_{fikt}$} \\ 
\cline{2-5}
            
\\[-1.8ex]
            &\multicolumn{2}{c}{SOE}&\multicolumn{2}{c}{PRIVATE}\\
\\[-1.8ex] & (1) & (2) & (3) & (4)\\
 \\[-1.8ex]&  $\text{Coastal}_c$  & $\text{NO  Coastal}_c$  &  $\text{Coastal}_c$  & $\text{NO  Coastal}_c$ \\
 \hline \\[-1.8ex] 
   $target_c \times \text{Period} \times \text{Polluted}_i$  & 0.158$^{**}$ & 0.119 & $-$0.012 & $-$0.087$^{**}$ \\ 
  & (0.063) & (0.098) & (0.023) & (0.036) \\ 
 \hline \\[-1.8ex] 
Firm & Yes & Yes & Yes & Yes \\ 
City-industry & Yes & Yes & Yes & Yes \\ 
City-time & Yes & Yes & Yes & Yes \\ 
time-industry & Yes & Yes & Yes & Yes \\ 
Observations & 19,540 & 21,948 & 477,084 & 130,225 \\ 
R$^{2}$ & 0.955 & 0.956 & 0.857 & 0.878 \\ 

\bottomrule 
\\ %%% Create second table
        \multicolumn{1}{l}{\textbf{Panel C: industrial concentration}} \\
        \toprule
         & \multicolumn{4}{c}{Dependent variable $\text { TFP }_{fikt}$} \\ 
\cline{2-5}
            
\\[-1.8ex]
            &\multicolumn{2}{c}{SOE}&\multicolumn{2}{c}{PRIVATE}\\
\\[-1.8ex] & (1) & (2) & (3) & (4)\\
 \\[-1.8ex]& Concentrated & NO Concentrated & Concentrated & NO Concentrated\\
 \hline \\[-1.8ex] 
   $target_c \times \text{Period} \times \text{Polluted}_i$  & 0.068 & 0.159$^{**}$ & $-$0.035 & $-$0.015 \\ 
  & (0.084) & (0.063) & (0.032) & (0.024) \\ 
 \hline \\[-1.8ex] 
Firm & Yes & Yes & Yes & Yes \\ 
City-industry & Yes & Yes & Yes & Yes \\ 
City-time & Yes & Yes & Yes & Yes \\ 
time-industry & Yes & Yes & Yes & Yes \\ 
Observations & 23,054 & 18,434 & 170,305 & 437,004 \\ 
R$^{2}$ & 0.957 & 0.953 & 0.869 & 0.859 \\ 
    \end{tabular}
    \end{adjustbox}
    \begin{tablenotes}
      \small
      \item 
      Note: $^{*}$p$<$0.1 $^{**}$p$<$0.05 $^{***}$p$<$0.01 \\
      Heteroskedasticity-robust standard errors in parentheses are clustered by city
    \end{tablenotes}
\end{table}





Estimates are reported in tables \ref{table_10} and \ref{table_11}. In the sub-samples of SOEs firms located in TCZ cities, the coefficient of interest is 0.144, positive and significant at 1$\%$, confirming the Porter hypothesis. For SOEs in non-TCZ cities, and private firms in TCZ cities, it is not significant, therefore the regulation has no effect on technological improvement. Finally, it is negative and significant for private firms in non-TCZ cities, suggesting that for those firms, the cost of the policy impedes the improvement of productivity. Similar findings hold for coastal (no-coastal) firms, with SOEs (private) in coastal (non-coastal) firms being positively (adversely) affected by the environmental regulation: the coefficient for SOEs in coastal areas is set at 0.158, while for private firms in no-coastal areas, it is - 0.087. The level of concentration matters as well, as reflected by the coefficient for SOEs firms located in cities where we consider an Herfindahl index below the 60th decile: 0.159 which is significant at 1$\%$ \footnote{Similar results hold for the 70th and 80th deciles, they are available upon request}. It confirms that smaller firms are more likely to invest in greener technologies which are usually riskier. Table \ref{table_11} confirms that for SOEs located in cities where GDP per capita is sufficiently high (above the turning points), the demand for a better environment and for a cleaner model of production translates into a significant and positive reaction to the regulation. In other cities this result does not hold anymore. 


Overall the results suggest that the policy-induced technological improvement hold only for SOEs, located in TCZ cities, in wealthier cities, and to a lesser extent in cities where the level of industrial concentration is lower. Therefore, the weaker policy-induced decrease in pollution that is reported in Section \ref{analysis} for cities where the share of SOEs is higher does not seem to be driven by an intrinsically smaller effort in technological improvement. It is related to their larger size, to their localization in poorer cities (according to the Kuznets definition) and non coastal areas, where the pressure from the demand side is lower and does not translate into a cleaner model of production. Finally, if the environmental policy-induced technological improvement and concomitant decrease in the emission of SO2 happens in the richest areas of the country, we cannot exclude this improvement to be due to companies adjusting to the regulation not only by improving their technology, as suggested by our calculus, but by physically (re)locating to provinces with lower environmental targets or weakest enforcement. The evidence about the pollution haven hypothesis in China is mixed: \cite{Wang2019-ju} do not support the pollution haven hypothesis in domestic trade during 2007–2012, while China seems to be $``$pollution heaven$"$ in South-South trade according to \cite{Lin2019-ft} or \cite{Sun2017-la}.

\begin{table}[!htb] \centering 
  \caption{\\ TFP: below and above turning points} 
\label{table_11}
\begin{adjustbox}{width=\textwidth, totalheight=\textheight-2\baselineskip,keepaspectratio}
\begin{tabular}{@{\extracolsep{5pt}}lccccc} 
\\[-1.8ex]\hline 
\hline \\[-1.8ex] 
 & \multicolumn{4}{c}{Dependent variable $\text { TFP }_{fikt}$} \\ 
\cline{2-6}
            
\\[-1.8ex]
            &\multicolumn{2}{c}{SOE}&\multicolumn{2}{c}{PRIVATE}\\
\\[-1.8ex] & (1) & (2) & (3) & (4) \\
 \\[-1.8ex]& Above & Below & Above & Below \\
 \hline \\[-1.8ex] 
$target_c \times \text{Period} \times \text{Polluted}_i$  &0.047 (0.102) & 0.024 (0.106) & -0.009 (0.020)& -0.008 (0.048) &  \\
Observations  & 5,130 & 35,622 & 186,699 & 413,836&  Column (4): $\text{No Concentrated}^a$\\
R$^{2}$  & 0.979&0.956&0.905&0.876&  RMB 45396\\
\hline 
$target_c \times \text{Period} \times \text{Polluted}_i$  &  0.117 (0.071) & 0.042 (0.116) & -0.018 (0.020) & -0.109 (0.069)&  \\
Observations  & 9.397&31,355 & 284,694 & 315,841 &  Column (10): $\text{SOE No dominated}^a$ \\ 
R$^{2}$  &0.971&0.960&0.890&0.883&  RMB 35190\\
\hline 
$target_c \times \text{Period} \times \text{Polluted}_i$  &  $0.168^{***}$  (0.058) & 0.108 (0.126)  &-0.013 (0.020) & 0.124 (0.076)& \\
Observations  & 12,605 & 28,147 & 350,740 & 249,795 &  Column (6): $\text{SOE No dominated}^a$ \\ 
R$^{2}$  & 0.965&0.963&0.882&0.892 &  RMB 30264\\
\hline 
$target_c \times \text{Period} \times \text{Polluted}_i$  & 
0.156$^{***}$ (0.056)& 0.178 (0.137)& $-$0.017  (0.021)& $-$0.146$^{*}$ (0.082) & \\
Observations  & 13,935 & 26,817 & 366,289 & 234,246 & Column (1): $TCZ^a$\\ 
R$^{2}$  &0.964&0.964&0.879&0.893&  RMB  28795\\ 
\hline 
$target_c \times \text{Period} \times \text{Polluted}_i$  & $0.123^{**}$ (0.051) & 0.130 (0.143) & -0.017 (0.021)&  -0.161 (0.100)& \\
Observations  & 18,061&22,691 & 419,579 & 180,956&  Column (8): $\text{SOE No dominated}^a$ \\ 
R$^{2}$  & 0.958&0.966&0.869&0.894&  RMB 24867\\
\hline \\[-1.8ex] 
\end{tabular}
\end{adjustbox}
\begin{tablenotes} 
 \small 
 \item \footnotesize{
The columns Above (Below) refer to firms in cities whose gdp per capita are strictly above (below) the Kuznets turning points. References for the latter are provided in the last column. \\
$a$ refers to the number of the column in table \ref{table_9}, which provides us with the estimated turning point. For instance 45 396 is estimated using the sub-sample of firms in no-concentrated cities, table \ref{table_9}, column 4. \\
Due to limited space, only the coefficients of interest are presented $^{*}$ Significance at the 10\%, $^{**}$ Significance at the 5\%, $^{***}$ Significance at the 1\%. Heteroscedasticity-robust standard errors in parentheses are clustered by city 
}
\end{tablenotes}
\end{table}
\section{Conclusion} \label{conclusion} 
\addcontentsline{toc}{section}{Conclusion}



This paper examines the effect of the change in environmental protection regime on firms' performance in China. This change consisted in a switch from a top-down to a bottom-up approach in 2006 and a new emphasis put on local incentives and target-based policy. We compute the policy-induced reduction of SO2 emission at the city level and distinguish TCZ (no TCZ), rich (poor) areas, cities where the level of industrial concentration is below (above) a given threshold, and SOEs (private sector) dominated cities. The findings suggest that pollution-intensive cities reaction is due to TCZ, rich and with a lower industrial concentration cities. More importantly for the purpose of this paper, SOEs dominated cities did not decreased their SO2 emission in response to the environmental regulation. SOEs dominated cities are mostly non TCZ cities, with a high level of industrial concentration, and they are also predominantly located in poorer areas of China. 


The empirical analysis is rooted in a unique and rich dataset provided by the Ministry of Environmental Protection (MEP) and by the State Environmental Protection Agency (SEPA), which collect the main data source of pollutants and wastes in China since 1980. The double difference in difference identification strategy allows us to quantify the effect of the environmental regulation on firms' emissions of pollution. 


Further investigation emphasizes several mechanisms at work to explain this absence of reaction of SOEs-dominated cities. SOEs are predominantly located in non-TCZ and relatively poorer areas (hinterlands). Kuznets curves confirm that the environmental regulation is less effective in those poorer areas with a larger presence of SOEs, while it contributes to environmental improvement, when middle income cities grow wealthier. Size matters as well: cities characterized by a larger industrial concentration do not reduce their emission of SO2 because the environmentally-induced budget constraint hardening does not constrain them and, yet, they benefit from a SBC. 


Finally we scrutinize the policy-induced firms TFP improvement by controlling for the heterogeneity of the cities' responses to the environmental regulation, and we find that SOEs are improving their productivity to adjust to the environmental targets under certain circumstances, when they are located in TCZ, in relatively wealthier areas and above the Kuznets turning points. These results are robust to various specifications and inclusions of city-year, industry-year and city-industry fixed effects. The analysis of TFP is realized with the inclusion of firms fixed effects. Besides, we document a slightly negative effect of the environmental protection for certain private enterprises, which does not for the sub-samples of no TCZ and no Coastal cities, an outcome which may be unique to developing countries, as emphasized in the literature (see \cite{Jefferson2013-az}). Finally, we cannot exclude that companies can adjust to the regulation not only by improving their technology, but by physically (re)locating to the provinces with lower environmental targets or weaker enforcement.

%\clearpage
%\bibliographystyle{plainnat}
%\bibliography{Bibliography/SBC_bibliography.bib}
\bibliography{Bibliography/reference.bib}

%\clearpage

\section{Appendix} \label{appendix}
\addcontentsline{toc}{section}{Appendix}

\begin{table}[!htb] \centering
  \caption{TCZ and SPZ cities in China}
  \begin{adjustbox}{width=\textwidth, totalheight=\textheight-2\baselineskip,keepaspectratio}
    \label{tab:appendix1}
\begin{tabular}{llllllllll}
\hline
Province  & City      & Code & TCZ & SPZ & Province     & City          & Code & TCZ & SPZ \\
\hline
Anhui     & Hefei     & 3401 & 0   & 0   & Guangdong    & Shanwei       & 4415 & 1   & 1   \\
Anhui     & Wuhu      & 3402 & 1   & 0   & Guangdong    & Heyuan        & 4416 & 0   & 1   \\
Anhui     & Bengbu    & 3403 & 0   & 0   & Guangdong    & Yangjiang     & 4417 & 0   & 1   \\
Anhui     & Huainan   & 3404 & 0   & 0   & Guangdong    & Qingyuan      & 4418 & 1   & 1   \\
Anhui     & Maanshan  & 3405 & 1   & 0   & Guangdong    & Dongguan      & 4419 & 1   & 1   \\
Anhui     & Huaibei   & 3406 & 0   & 0   & Guangdong    & Zhongshan     & 4420 & 1   & 1   \\
Anhui     & Tongling  & 3407 & 1   & 0   & Guangdong    & Chaozhou      & 4421 & 1   & 1   \\
Anhui     & Anqing    & 3408 & 0   & 0   & Guangdong    & Jieyang       & 4424 & 1   & 1   \\
Anhui     & Huangshan & 3409 & 1   & 0   & Guangxi      & Nanning       & 4501 & 1   & 1   \\
Anhui     & Fuyang    & 3412 & 0   & 0   & Guangxi      & Liuzhou       & 4502 & 1   & 1   \\
Anhui     & Liuan     & 3415 & 0   & 0   & Guangxi      & Guilin        & 4503 & 1   & 1   \\
Anhui     & Xuancheng & 3418 & 1   & 0   & Guangxi      & Wuzhou        & 4504 & 1   & 1   \\
Anhui     & Chizhou   & 3417 & 0   & 0   & Guangxi      & Beihai        & 4505 & 0   & 1   \\
Beijing   & Beijing   & 1101 & 1   & 1   & Guangxi      & Yulin         & 4506 & 1   & 1   \\
Chongqing & Chongqing & 5001 & 1   & 1   & Guangxi      & Baise         & 4510 & 0   & 1   \\
Fujian    & Fuzhou    & 3501 & 1   & 1   & Guangxi      & Hechi         & 4508 & 1   & 1   \\
Fujian    & Xiamen    & 3502 & 1   & 1   & Guangxi      & Qinzhou       & 4509 & 0   & 1   \\
Fujian    & Putian    & 3503 & 0   & 1   & Guangxi      & Fangchenggang & 4512 & 0   & 1   \\
Fujian    & Sanming   & 3504 & 1   & 1   & Guangxi      & Guigang       & 4513 & 1   & 1   \\
Fujian    & Quanzhou  & 3505 & 1   & 1   & Guangxi      & Hezhou        & 4516 & 1   & 1   \\
Fujian    & Zhangzhou & 3506 & 1   & 1   & Guizhou      & Guiyang       & 5201 & 1   & 1   \\
Fujian    & Nanping   & 3507 & 0   & 1   & Guizhou      & Liupanshui    & 5202 & 0   & 1   \\
Fujian    & Ningde    & 3508 & 0   & 1   & Guizhou      & Zunyi         & 5203 & 1   & 1   \\
Fujian    & Longyan   & 3509 & 1   & 1   & Guizhou      & Anshun        & 5207 & 1   & 1   \\
Gansu     & Lanzhou   & 6201 & 1   & 1   & Hainan       & Haikou        & 4601 & 0   & 0   \\
Gansu     & Jiayuguan & 6202 & 0   & 1   & Hebei        & Shijiazhuang  & 1301 & 1   & 1   \\
Gansu     & Jinchang  & 6203 & 1   & 1   & Hebei        & Tangshan      & 1302 & 1   & 1   \\
Gansu     & Baiyin    & 6204 & 1   & 1   & Hebei        & Qinhuangdao   & 1303 & 0   & 1   \\
Gansu     & Tianshui  & 6205 & 0   & 1   & Hebei        & Handan        & 1304 & 1   & 1   \\
Gansu     & Jiuquan   & 6206 & 0   & 1   & Hebei        & Xingtai       & 1305 & 1   & 1   \\
Gansu     & Zhangye   & 6207 & 1   & 1   & Hebei        & Baoding       & 1306 & 1   & 1   \\
Gansu     & Wuwei     & 6208 & 0   & 1   & Hebei        & Zhangjiakou   & 1307 & 1   & 1   \\
Gansu     & Dingxin   & 6209 & 0   & 1   & Hebei        & Chengde       & 1308 & 1   & 1   \\
Gansu     & Longnann  & 6210 & 0   & 1   & Hebei        & Cangzhou      & 1309 & 0   & 1   \\
Gansu     & Pingliang & 6211 & 0   & 1   & Hebei        & Langfang      & 1310 & 0   & 1   \\
Gansu     & Qingyangn & 6212 & 0   & 1   & Hebei        & Hengshui      & 1311 & 1   & 1   \\
Guangdong & Guangzhou & 4401 & 1   & 1   & Heilongjiang & Harbin        & 2301 & 0   & 0   \\
Guangdong & Shaoguan  & 4402 & 1   & 1   & Heilongjiang & Qiqihar       & 2302 & 0   & 0   \\
Guangdong & Shenzhen  & 4403 & 1   & 1   & Heilongjiang & Jixi          & 2303 & 0   & 0   \\
Guangdong & Zhuhai    & 4404 & 1   & 1   & Heilongjiang & Hegang        & 2304 & 0   & 0   \\
Guangdong & Shantou   & 4405 & 1   & 1   & Heilongjiang & Shuangyashan  & 2305 & 0   & 0   \\
Guangdong & Foshan    & 4406 & 1   & 1   & Heilongjiang & Daqing        & 2306 & 0   & 0   \\
Guangdong & Jiangmen  & 4407 & 1   & 1   & Heilongjiang & Yichun        & 2307 & 0   & 0   \\
Guangdong & Zhanjiang & 4408 & 1   & 1   & Heilongjiang & Jiamusi       & 2308 & 0   & 0   \\
Guangdong & Maoming   & 4409 & 0   & 1   & Heilongjiang & Qitaihe       & 2309 & 0   & 0   \\
Guangdong & Zhaoqing  & 4412 & 1   & 1   & Heilongjiang & Mudanjiang    & 2310 & 0   & 0   \\
Guangdong & Huizhou   & 4413 & 1   & 1   & Heilongjiang & Heihe         & 2311 & 0   & 0   \\
Guangdong & Meizhou   & 4414 & 0   & 1   & Heilongjiang & Suihua        & 2314 & 0   & 0  
\end{tabular}
\end{adjustbox}
\end{table}

\begin{table}[!htb] \centering
  \caption{TCZ and SPZ cities in China (continued)}
  \begin{adjustbox}{width=\textwidth, totalheight=\textheight-2\baselineskip,keepaspectratio}
    \label{tab:appendix2}
\begin{tabular}{llllllllll}
\hline
Province       & City         & Code & TCZ & SPZ & Province & City        & Code & TCZ & SPZ \\
\hline
Henan          & Zhengzhou    & 4101           & 1   & 1   & Jiangsu  & Xuzhou      & 3203           & 1   & 1   \\
Henan          & Kaifeng      & 4102           & 0   & 1   & Jiangsu  & Changzhou   & 3204           & 1   & 1   \\
Henan          & Luoyang      & 4103           & 1   & 1   & Jiangsu  & Suzhou      & 3205           & 1   & 1   \\
Henan          & Pingdingshan & 4104           & 0   & 1   & Jiangsu  & Nantong     & 3206           & 1   & 1   \\
Henan          & Anyang       & 4105           & 1   & 1   & Jiangsu  & Lianyungang & 3207           & 0   & 1   \\
Henan          & Hebi         & 4106           & 0   & 1   & Jiangsu  & Yancheng    & 3209           & 0   & 1   \\
Henan          & Xinxiang     & 4107           & 0   & 1   & Jiangsu  & Yangzhou    & 3210           & 1   & 1   \\
Henan          & Jiaozuo      & 4108           & 1   & 1   & Jiangsu  & Zhenjiang   & 3211           & 1   & 1   \\
Henan          & Puyang       & 4109           & 0   & 1   & Jiangsu  & Taizhou     & 3212           & 1   & 1   \\
Henan          & Xuchang      & 4110           & 0   & 1   & Jiangsu  & Suqian      & 3217           & 0   & 1   \\
Henan          & Luohe        & 4111           & 0   & 1   & Jiangsu  & Huaian      & 3221           & 0   & 1   \\
Henan          & Sanmenxia    & 4112           & 1   & 1   & Jiangxi  & Nanchang    & 3601           & 1   & 1   \\
Henan          & Shangqiu     & 4113           & 0   & 1   & Jiangxi  & Jingdezhen  & 3602           & 0   & 1   \\
Henan          & Zhoukou      & 4114           & 0   & 1   & Jiangxi  & Pingxiang   & 3603           & 1   & 1   \\
Henan          & Zhumadian    & 4115           & 0   & 1   & Jiangxi  & Jiujiang    & 3604           & 1   & 1   \\
Henan          & Nanyang      & 4116           & 0   & 1   & Jiangxi  & Xinyu       & 3605           & 0   & 1   \\
Henan          & Xinyangn     & 4117           & 0   & 1   & Jiangxi  & Yingtan     & 3606           & 1   & 1   \\
Hubei          & Wuhan        & 4201           & 1   & 1   & Jiangxi  & Ganzhoun    & 3607           & 1   & 1   \\
Hubei          & Huangshi     & 4202           & 1   & 1   & Jiangxi  & Yichun      & 3608           & 0   & 1   \\
Hubei          & Shiyan       & 4203           & 0   & 1   & Jiangxi  & Shangrao    & 3609           & 0   & 1   \\
Hubei          & Yichang      & 4205           & 1   & 1   & Jiangxi  & Jian        & 3610           & 1   & 1   \\
Hubei          & Xiangfan     & 4206           & 0   & 1   & Jiangxi  & Fuzhou      & 3611           & 1   & 1   \\
Hubei          & Ezhou        & 4207           & 1   & 1   & Jilin    & Changchun   & 2201           & 0   & 0   \\
Hubei          & Jingmen      & 4208           & 1   & 1   & Jilin    & Jilin       & 2202           & 1   & 0   \\
Hubei          & Huanggang    & 4209           & 0   & 1   & Jilin    & Siping      & 2203           & 1   & 0   \\
Hubei          & Xiaogan      & 4210           & 0   & 1   & Jilin    & Liaoyuan    & 2204           & 0   & 0   \\
Hubei          & Xianning     & 4211           & 1   & 1   & Jilin    & Tonghua     & 2205           & 1   & 0   \\
Hubei          & Jingzhou     & 4212           & 1   & 1   & Jilin    & Baicheng    & 2209           & 0   & 0   \\
Hubei          & Suizhou      & 4215           & 0   & 1   & Liaoning & Shenyang    & 2101           & 1   & 1   \\
Hunan          & Changsha     & 4301           & 1   & 1   & Liaoning & Dalian      & 2102           & 1   & 1   \\
Hunan          & Zhuzhou      & 4302           & 1   & 1   & Liaoning & Anshan      & 2103           & 1   & 1   \\
Hunan          & Xiangtan     & 4303           & 1   & 1   & Liaoning & Fushun      & 2104           & 1   & 1   \\
Hunan          & Hengyang     & 4304           & 1   & 1   & Liaoning & Benxi       & 2105           & 1   & 1   \\
Hunan          & Shaoyang     & 4305           & 0   & 1   & Liaoning & Dandong     & 2106           & 0   & 1   \\
Hunan          & Yueyang      & 4306           & 1   & 1   & Liaoning & Jinzhou     & 2107           & 1   & 1   \\
Hunan          & Changde      & 4307           & 1   & 1   & Liaoning & Yingkou     & 2108           & 0   & 1   \\
Hunan          & Yiyang       & 4309           & 1   & 1   & Liaoning & Fuxin       & 2109           & 1   & 1   \\
Hunan          & Loudin       & 4310           & 1   & 1   & Liaoning & Liaoyang    & 2110           & 1   & 1   \\
Hunan          & Chenzhou     & 4311           & 1   & 1   & Liaoning & Panjin      & 2111           & 0   & 1   \\
Hunan          & Huaihua      & 4312           & 1   & 1   & Liaoning & Tieling     & 2112           & 0   & 1   \\
Inner Mongolia & Hohhot       & 1501           & 1   & 1   & Liaoning & Chaoyang    & 2113           & 0   & 1   \\
Inner Mongolia & Baotou       & 1502           & 1   & 1   & Ningxia  & Yinchuan    & 6401           & 1   & 1   \\
Inner Mongolia & Wuhai        & 1503           & 1   & 1   & Ningxia  & Shizuishan  & 6402           & 1   & 1   \\
Inner Mongolia & Chifeng      & 1504           & 1   & 1   & Ningxia  & Guyuann     & 6404           & 0   & 1   \\
Inner Mongolia & Hulunbeirn   & 1507           & 0   & 1   & Qinghai  & Xining      & 6301           & 0   & 0   \\
Inner Mongolia & Ulanqabn     & 1510           & 0   & 1   & Shaanxi  & Xian        & 6101           & 1   & 1   \\
Inner Mongolia & Bayannaoern  & 1511           & 0   & 1   & Shaanxi  & Tongchuan   & 6102           & 1   & 1   \\
Jiangsu        & Nanjing      & 3201           & 1   & 1   & Shaanxi  & Baoji       & 6103           & 0   & 1   \\
Jiangsu        & Wuxi         & 3202           & 1   & 1   & Shaanxi  & Xianyang    & 6104           & 0   & 1  
\end{tabular}
\end{adjustbox}
\end{table}

\begin{table}[!htb] \centering
  \caption{TCZ and SPZ cities in China (continued)}
  \begin{adjustbox}{width=\textwidth, totalheight=\textheight-2\baselineskip,keepaspectratio}
    \label{tab:appendix3}
\begin{tabular}{llllllllll}
\hline
Province       & City         & Code & TCZ & SPZ & Province & City        & Code & TCZ & SPZ \\
\hline
Shaanxi  & Weinan    & 6105           & 1   & 1   & Xinjiang & Karamay  & 6502           & 0   & 1   \\
Shaanxi  & Hanzhong  & 6106           & 0   & 1   & Yunnan   & Kunming  & 5301           & 1   & 1   \\
Shaanxi  & Ankang    & 6107           & 0   & 1   & Yunnan   & Zhaotong & 5306           & 1   & 1   \\
Shaanxi  & Shangluon & 6108           & 1   & 1   & Yunnan   & Qujing   & 5303           & 1   & 1   \\
Shaanxi  & Yanan     & 6109           & 0   & 1   & Yunnan   & Simaon   & 5309           & 0   & 1   \\
Shaanxi  & Yulinn    & 6110           & 0   & 1   & Yunnan   & Baoshan  & 5312           & 0   & 1   \\
Shandong & Jinan     & 3701           & 1   & 1   & Yunnan   & Lijiangn & 5314           & 0   & 1   \\
Shandong & Qingdao   & 3702           & 1   & 1   & Yunnan   & Lincangn & 5317           & 0   & 1   \\
Shandong & Zibo      & 3703           & 1   & 1   & Zhejiang & Hangzhou & 3301           & 1   & 1   \\
Shandong & Zaozhuang & 3704           & 1   & 1   & Zhejiang & Ningbo   & 3302           & 1   & 1   \\
Shandong & Dongying  & 3705           & 0   & 1   & Zhejiang & Wenzhou  & 3303           & 1   & 1   \\
Shandong & Yantai    & 3706           & 1   & 1   & Zhejiang & Jiaxing  & 3304           & 1   & 1   \\
Shandong & Weifang   & 3707           & 1   & 1   & Zhejiang & Huzhou   & 3305           & 1   & 1   \\
Shandong & Jining    & 3708           & 1   & 1   & Zhejiang & Shaoxing & 3306           & 1   & 1   \\
Shandong & Taian     & 3709           & 1   & 1   & Zhejiang & Jinhua   & 3307           & 1   & 1   \\
Shandong & Weihai    & 3710           & 0   & 1   & Zhejiang & Quzhou   & 3308           & 1   & 1   \\
Shandong & Rizhao    & 3711           & 0   & 1   & Zhejiang & Zhoushan & 3309           & 0   & 1   \\
Shandong & Liaocheng & 3714           & 0   & 1   & Zhejiang & Lishui   & 3310           & 0   & 1   \\
Shandong & Linyi     & 3713           & 0   & 1   & Zhejiang & Taizhou  & 3311           & 1   & 1   \\
Shandong & Heze      & 3717           & 0   & 1   &          &          &                &     &     \\
Shandong & Laiwu     & 3720           & 1   & 1   &          &          &                &     &     \\
Shanghai & Shanghai  & 3101           & 1   & 1   &          &          &                &     &     \\
Shanxi   & Taiyuan   & 1401           & 1   & 1   &          &          &                &     &     \\
Shanxi   & Datong    & 1402           & 1   & 1   &          &          &                &     &     \\
Shanxi   & Yangquan  & 1403           & 1   & 1   &          &          &                &     &     \\
Shanxi   & Changzhi  & 1404           & 0   & 1   &          &          &                &     &     \\
Shanxi   & Jincheng  & 1405           & 0   & 1   &          &          &                &     &     \\
Shanxi   & Shuozhou  & 1406           & 1   & 1   &          &          &                &     &     \\
Shanxi   & Xinzhou   & 1408           & 1   & 1   &          &          &                &     &     \\
Shanxi   & Luliangn  & 1409           & 0   & 1   &          &          &                &     &     \\
Shanxi   & Jinzhong  & 1417           & 1   & 1   &          &          &                &     &     \\
Shanxi   & Linfen    & 1410           & 1   & 1   &          &          &                &     &     \\
Shanxi   & Yuncheng  & 1412           & 1   & 1   &          &          &                &     &     \\
Sichuan  & Chengdu   & 5101           & 1   & 1   &          &          &                &     &     \\
Sichuan  & Zigong    & 5103           & 1   & 1   &          &          &                &     &     \\
Sichuan  & Panzhihua & 5104           & 1   & 1   &          &          &                &     &     \\
Sichuan  & Luzhou    & 5105           & 1   & 1   &          &          &                &     &     \\
Sichuan  & Deyang    & 5106           & 1   & 1   &          &          &                &     &     \\
Sichuan  & Mianyang  & 5107           & 1   & 1   &          &          &                &     &     \\
Sichuan  & Guangyuan & 5108           & 0   & 1   &          &          &                &     &     \\
Sichuan  & Suining   & 5109           & 1   & 1   &          &          &                &     &     \\
Sichuan  & Neijiang  & 5110           & 1   & 1   &          &          &                &     &     \\
Sichuan  & Leshan    & 5111           & 1   & 1   &          &          &                &     &     \\
Sichuan  & Yibin     & 5114           & 1   & 1   &          &          &                &     &     \\
Sichuan  & Nanchong  & 5113           & 1   & 1   &          &          &                &     &     \\
Sichuan  & Yaan      & 5118           & 0   & 1   &          &          &                &     &     \\
Sichuan  & Guangan   & 5122           & 1   & 1   &          &          &                &     &     \\
Tianjin  & Tianjin   & 1201           & 1   & 1   &          &          &                &     &     \\
Xinjiang & Urumqi    & 6501           & 1   & 1   &          &          &                &     &    
\end{tabular}
\end{adjustbox}
\end{table}


\end{document}
