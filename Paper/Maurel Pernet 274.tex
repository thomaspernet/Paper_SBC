\documentclass[12pt]{article}

\begin{document}

\subsection{Policy variables: TCZ and target}
\addcontentsline{toc}{subsection}{Policy variables: TCZ and Target}


As documented in the Environmental Policy Background section \ref{policy}, the State Council launched in 1998 a vast policy to curb SO2 emissions and to reduce the acid rain. There were 175 cities called TCZ cities located in 27 provinces designated to provide the subsequent effort for controlling SO2 emissions. Out of the 270 cities in our data set, 140 are qualified as TCZ cities. Table \ref{appendix} in the appendix provides the list of TCZ cities present in our data set. $TCZ_{i}$ is a dummy set equal to one if city $i$ belongs to this list. 




\begin{table}[!htb] \centering
    \caption{\\ GDP per capita, population and SO2 emissions}
      \begin{adjustbox}{width=\textwidth, totalheight=\textheight-2\baselineskip,keepaspectratio}
    \label{table_3}
    \begin{tabular}{lrrrrrrrrrrrr}
      \multicolumn{1}{l}{\textbf{\small Panel A:}} \\
      \multicolumn{1}{l}{\textbf{\small GDP per capita and population}} \\
      \toprule
     & \multicolumn{3}{c}{No TCZ} & \multicolumn{3}{c}{TCZ} \\
      & (1)  & (2) & & (3)  & (4) \\
      & 2004-2005 &  2006-2007 & & 2004-2005 &2006-2007  \\
      
      \midrule
      $\text{gdp per capita}_i$    & 14,218 & 19,901 & & 23,958 & 32,861 \\
$\text{population}_i$ &     81 &     84 & &   155 &    163  \\

      \bottomrule
      \\ %%% Create second table
        \multicolumn{1}{l}{\textbf{Panel B:}} \\
        \multicolumn{1}{l}{\textbf{SO2 emissions (millions of tonnes)}} \\
        \toprule
       & \multicolumn{3}{c}{No TCZ} & \multicolumn{3}{c}{TCZ} \\
      & (1)  & (2) & (2) - (1) & (4)  & (5)  &(5) - (4) \\
      & 2004-2005 &  2006-2007 & &  2004-2005 &2006-2007  \\
\hline \\[-1.8ex] 
Full sample &  2.624  & 2.833  &  0.209 &  9.736 &10.294  &  0.558  \\
Central     &  0.991  & 1.004  &  0.013 & 2.138 & 2.154  &   0.016  \\
Coastal     &  0.556  & 0.632  &  0.076 & 3.763 & 3.860 &   0.097  \\
Northeast   &  0.354  & 0.389  &  0.035 & 0.537 & 0.736 &    0.199  \\
Northwest   &  0.265  &  0.360 &  0.096 & 1.110 & 1.097 &    -0.012  \\
Southwest   &  0.459  & 0.448  &  -0.010 & 2.189 &  2.448 &   0.259  \\
      \bottomrule
      \hline
    \end{tabular}
    \end{adjustbox}
    \begin{tablenotes}
      \small
      \item 
      \footnotesize{
      Sources: Authors' own computation \\
  Panel A: $\text{gdp per capita}_i$ is in RMB and $\text{population}_i$ is in million. \\ $\text{gdp per capita}_i$ and and $\text{population}_i$ are averaged over 2004-2005 and 2006-2007. They are borrowed from the China City Statistical Yearbooks 2002–2007. \\
  Panel B: reported numbers are in millions of tonnes.  \\
  All variables are summed over the years 2004 and 2005 and over the years 2006 and 2007. 
      }
    \end{tablenotes}
\end{table}

Table \ref{table_3} Panel A provides $\text{gdp per capita}_i$ and $\text{population}_i$ in TCZ (non-TCZ cities) and SOEs dominated (no SOEs dominated) cities. Panel B report SO2 Emissions. We notice that TCZ cities are, by definition, making more effort to execute stringent environmental regulations. We get also some preliminary evidence of a positive relationship between wealth and the demand for cleaner environmental goods. In the Chinese case, the evidence is documented in \cite{Hering2014-af}\footnote{In section \ref{analysis}, we estimate Kuznets curves to check that richest cities pollute less.}. This suggests that SOEs dominated cities being richer on average are therefore potentially more likely to fight against pollution  than their counterparts. Overall SO2 emissions averaged over 2006-2007 exceed the level in the period before, which suggests the existence of a trend and the need to control for it in the empirical analysis. 


Table \ref{table_3} Panel B gives a more in-depth overview of the SO2 pattern in the major areas of China. Following \cite{Wu2017-bl}, we split the cities between Coastal, Southwest, Central, Northeast, and Northwest areas\footnote{This province breakdown follows the paper of \cite{Wu2017-bl}. The Central provinces are Anhui Henan, Hubei, Hunan, Jiangxi, and Shanxi. The Coastal provinces are Beijing, Fujian, Guangdong, Hainan Hebei, Jiangsu, Shandong, Shanghai, Tianjin, and Zhejiang. The Northeastern provinces are Heilongjiang, Jilin, Liaoning. Northwest are Gansu, Inner Mongolia, Ningxia, Qinghai, Shaanxi, and Xinjiang. The Southwestern parts are Chongqing, Guangxi, Guizhou, Sichuan, Yunnan, and Xizang.}. In our sample, the Coastal area of China is composed of ten provinces and has a total of 68 TCZ cities. This area is the wealthiest part of China: it represents the lion's share of national production and attracts the most significant foreign investment flows. The Southwestern area has five provinces and 24 TCZ cities, while the Central area has six provinces and 38 TCZ cities. The Northern part of China is split into its Western area with six provinces and 13 TCZ cities and Eastern area with three provinces and 11 TCZ cities. 



\subsection{Target-based evaluation system}
\addcontentsline{toc}{subsection}{Target-based evaluation system}



In 2006, the central government provided a clear SO2 pollution reduction guideline for the provinces in China, called the target-based evaluation policy, which the government used to deepen its political ties with the local cadres and to guarantee the fulfillment of the pollution reduction targets. The document stipulated that the provincial leaders had a binding contract with the Ministry of Environment and would bear the responsibility for any failure. Since the guideline, reduction mandates and environmental target-based evaluation system are not available at the city $i$ level, but at the provincial level, we apply the following formula\footnote{Which we borrowed from \cite{Chen2018-bs}}. For $t$ = 2006 or 2007: 
\begin{equation} \label{eq:equation_2}
target_{it} = target_{i} = \Delta SO2_{i, 05 − 10}= \Delta SO2_{p, 05 − 10} \times \sum_{k=1}^{29} \mu_{k} \frac{\text Y_{ki,2005}} {Y_{kp,2005}}
\end{equation}

where $i$ stands for city, $p$ for province and $k$ for the two digit industry ($k$ varies from 1 to 29). The left-hand side of the formula $target_{it}$ evaluates by how much a city should reduce its SO2 emissions between 2005 and 2010, in 10,000 tonnes.   

$\Delta SO2_{p, 05 − 10}$ refers to the reduction mandate at the provincial level. It is available for the 31 provinces of China and over the period 2005-2010 \footnote{ For instance, the province of Shanghai is expected to reduce its SO2 emissions by 13,000 tonnes over the period 2005-2010.}.  $\mu_{k} \frac{\text Y_{ki,2005}} {Y_{kp,2005}}$ is the share of industrial production $k$, in city $i$, over the total output of industry $k$, in province $p$, multiplied by $\mu_{k}$. $\mu_{k}$ is a weight which reflects each $k$ industry's contribution to the total industrial SO2 emissions. It is set equal to the ratio of SO2 emission in industry $k$ over total SO2 emission. The information about pollution emitted at the two digit level is taken from the MEP data set. All values refer to 2005.

Table \ref{table_4} provides an overall picture of the efforts required on average by the Chinese cities. Not only TCZ cities, but also cities along the coastal area need to engage in more work to meet the requirement in term of SO2 reduction at the end of the 11th FYP. The majority of cities with a larger share of SOE have the obligation to reduce by a larger extent their emission of SO2.


\begin{table}[!htb] \centering
\caption{\\ Mean target (millions of tonnes) in SOEs dominated cities \textit{versus} no SOEs dominated cities}
\label{table_4}
\begin{adjustbox}{width=\textwidth, totalheight=\textheight-2\baselineskip,keepaspectratio}
\begin{tabular}{lrrr}
\toprule
{} & All Cities & no SOEs dominated & SOEs dominated \\
index                &            &                  &               \\
\midrule
Full sample          &   0.108646 &         0.101678 &      0.123812 \\
Central              &   0.083170 &         0.065356 &      0.116160 \\
Coastal              &   0.166539 &         0.168427 &      0.154353 \\
Northeast            &   0.053427 &         0.045074 &      0.069373 \\
Northwest            &   0.053444 &         0.035052 &      0.072919 \\
Southwest            &   0.129409 &         0.080275 &      0.194058 \\
Central              &   0.083170 &         0.065356 &      0.116160 \\
Eastern              &   0.129370 &         0.132723 &      0.114789 \\
Western              &   0.100256 &         0.064868 &      0.135644 \\
No TCZ               &   0.052559 &         0.043773 &      0.077367 \\
TCZ                  &   0.160727 &         0.164138 &      0.154775 \\
Concentrated city    &   0.083507 &         0.049193 &      0.122500 \\
No Concentrated city &   0.136124 &         0.137464 &      0.128369 \\
\bottomrule
\end{tabular}
\end{adjustbox}
\begin{tablenotes} 
 \small 
 \item 
Sources: Author's own computation \\
The list of TCZ is provided by the State Council, 1998. \\
(No) SOEs dominated cities refers to cities where the 
(output, capital, employment) share of SOEs is (below) above a critical threshold, for instance the 60th decile. (No) Concentrated city refers to cities where the Herfindahl index is (below) above a critical threshold, for instance the 60th decile.
\end{tablenotes}
\end{table}
%\end{sidewaystable}

\subsection{Control variables}
\addcontentsline{toc}{subsection}{Control variables}

The literature has listed the key determinants of environmental degradation at the firm level (\citealt{Cole2003-ad,Cole2008-pj}). Capital intensity affects both the emissions and intensity of pollution (\citealt{Hering2014-af,Andersen2017-wf}). Firms' size matters: large industries emit more pollutants. Besides, we use the NBS industrial classification to sort firms according to the sector they belong to. We rely on the 2002 four-digit CIC and compute total employment, total output, and total net fixed assets, aggregated at the city-industry-year level. The information is generated from the Annual Survey of Industrial Firms (ASIF) conducted by China's NBS for the period from 2002- 2007. 


\section{Empirical analysis} \label{analysis} 
\addcontentsline{toc}{section}{Empirical analysis}


\subsection{Main results}
\addcontentsline{toc}{subsection}{Main results}



Table \ref{table_5} (columns 1 to 8) reports the results of equation \ref{eq:equation_1}. The coefficient of interest measures the effect of the environmental target-based policy on the emissions of SO2 in the polluting sectors, with a particular emphasis on cities dominated by SOEs. The triple interaction term $(target_{i} \times \text { Polluted sectors }_{k} \times \text { Period })$ estimate is negative and significant at 5$\%$  (column 1) and 1$\%$  (column 2), meaning that SO2 emissions felt significantly after the launch of the 11th FYP, more in cities where the intensity of the treatment was higher, in line with our expectations. A straightforward calculus shows that the reduction of SO2 reaches about 5\% of the average emission of Polluted sectors after 2006 in no SOEs dominated cities \footnote{4.9\% = 1-exp(0.478*0.101678), with 0.101678 being the average $target_i$ value for no SOEs dominated cities, see table \ref{table_4}, column 3. Assuming that SOEs dominated cities, where $target_i$ is set equal to 0.12381 millions tonnes, would react the same way to the Target policy, SO2 emission in those cities could also decrease by 5\%.}. Other control variables have the expected signs: economic growth has severely degraded the environment; GDP, employment and fixed assets are correlated with more emissions of SO2. 




\begin{table}[!htb] \centering
  %\resizebox{1\textwidth}{!}{
    %\begin{threeparttable}
    \caption{\\ 
    Environmental regulation effectiveness in SOEs dominated cities \textit{versus} no SOEs dominated cities}
      \begin{adjustbox}{width=\textwidth, totalheight=\textheight-2\baselineskip,keepaspectratio}
     \label{table_5}
      \begin{tabular}{@{\extracolsep{5pt}}lcccccccc} 
        \multicolumn{1}{l}{\textbf{Panel A: SOEs dominated cities}} \\
        \toprule
        & \multicolumn{8}{c}{Dependent variable $\text { SO2 emission }_{i k t}$} \\ 
        \cline{2-9}
            
\\[-1.8ex]
            &\multicolumn{2}{c}{Full sample}&\multicolumn{2}{c}{Output}&\multicolumn{2}{c}{Capital}&\multicolumn{2}{c}{Employment}\\
\\[-1.8ex] & (1) & (2) & (3) & (4) & (5) & (6) & (7) & (8)\\ 
\hline \\[-1.8ex] 
  $output_{cit}$ & 0.018 & 0.040 & 0.864$^{***}$ & 0.017 & 0.983$^{***}$ & $-$0.040 & 0.857$^{***}$ & 0.020 \\ 
  & (0.087) & (0.042) & (0.150) & (0.090) & (0.284) & (0.085) & (0.167) & (0.091) \\ 
  $capital_{cit}$ & 1.576$^{***}$ & $-$0.024 & $-$2.643$^{***}$ & $-$0.465 & $-$3.386$^{***}$ & $-$0.444 & $-$2.958$^{***}$ & $-$0.494 \\ 
  & (0.582) & (0.173) & (0.656) & (0.416) & (0.835) & (0.443) & (0.771) & (0.421) \\ 
  $labour_{cit}$ & 2.769$^{***}$ & 0.246 & 11.075$^{***}$ & 1.268$^{*}$ & 13.226$^{***}$ & 1.536$^{*}$ & 12.084$^{***}$ & 1.378$^{*}$ \\ 
  & (0.852) & (0.167) & (1.942) & (0.732) & (2.190) & (0.849) & (2.184) & (0.768) \\ 
   $target_c \times \text{Period}$  & 0.088 &  & 0.059 &  & $-$0.128 &  & 0.147 &  \\ 
  & (0.109) &   & (0.335) &   & (0.324) &   & (0.324) &   \\ 
   $target_c \times \text{Polluted}_i$  & 0.650$^{***}$ &  & 1.055$^{***}$ &  & 0.829$^{**}$ &  & 0.998$^{***}$ &  \\ 
  & (0.150) &   & (0.340) &   & (0.350) &   & (0.330) &   \\ 
   $target_c \times \text{Period} \times \text{Polluted}_i$  & $-$0.352$^{**}$ & $-$0.478$^{***}$ & $-$0.360 & 0.110 & $-$0.059 & 0.137 & $-$0.334 & $-$0.168 \\ 
  & (0.156) & (0.146) & (0.442) & (0.448) & (0.409) & (0.407) & (0.400) & (0.418) \\ 
 \hline \\[-1.8ex] 
City fixed effects & Yes & No & Yes & No & Yes & No & Yes & No \\ 
Industry fixed effects & Yes & No & Yes & No & Yes & No & Yes & No \\ 
Year fixed effects & Yes & No & Yes & No & Yes & No & Yes & No \\ 
City-year fixed effects & No & Yes & No & Yes & No & Yes & No & Yes \\ 
Industry-year fixed effects & No & Yes & No & Yes & No & Yes & No & Yes \\ 
City-industry fixed effects & No & Yes & No & Yes & No & Yes & No & Yes \\ 
Observations & 61,297 & 61,297 & 18,381 & 18,381 & 18,367 & 18,367 & 18,350 & 18,350 \\ 
R$^{2}$ & 0.432 & 0.878 & 0.470 & 0.892 & 0.461 & 0.887 & 0.474 & 0.893 \\ 
        \bottomrule
        \\ %%% Create second table
        \multicolumn{1}{l}{\textbf{Panel B: no SOEs dominated cities}} \\
        \toprule
        & \multicolumn{8}{c}{Dependent variable $\text { SO2 emission }_{i k t}$} \\ 
        \cline{2-9}
            
\\[-1.8ex]
            &\multicolumn{2}{c}{Full sample}&\multicolumn{2}{c}{Output}&\multicolumn{2}{c}{Capital}&\multicolumn{2}{c}{Employment}\\
\\[-1.8ex] & (1) & (2) & (3) & (4) & (5) & (6) & (7) & (8)\\ 
\hline \\[-1.8ex] 
  $output_{cit}$ & 0.018 & 0.040 & $-$0.167$^{*}$ & 0.024 & $-$0.141$^{*}$ & 0.020 & $-$0.153$^{*}$ & 0.024 \\ 
  & (0.087) & (0.042) & (0.099) & (0.043) & (0.080) & (0.040) & (0.090) & (0.043) \\ 
  $capital_{cit}$ & 1.576$^{***}$ & $-$0.024 & 2.846$^{***}$ & 0.292 & 2.950$^{***}$ & 0.297 & 2.729$^{***}$ & 0.287 \\ 
  & (0.582) & (0.173) & (0.457) & (0.219) & (0.472) & (0.213) & (0.448) & (0.216) \\ 
  $labour_{cit}$ & 2.769$^{***}$ & 0.246 & 2.319$^{***}$ & 0.087 & 2.230$^{***}$ & 0.094 & 2.199$^{***}$ & 0.089 \\ 
  & (0.852) & (0.167) & (0.706) & (0.141) & (0.672) & (0.133) & (0.698) & (0.140) \\ 
   $target_c \times \text{Period}$  & 0.088 &  & 0.058 &  & 0.055 &  & 0.036 &  \\ 
  & (0.109) &   & (0.112) &   & (0.114) &   & (0.113) &   \\ 
   $target_c \times \text{Polluted}_i$  & 0.650$^{***}$ &  & 0.580$^{***}$ &  & 0.581$^{***}$ &  & 0.573$^{***}$ &  \\ 
  & (0.150) &   & (0.160) &   & (0.159) &   & (0.159) &   \\ 
   $target_c \times \text{Period} \times \text{Polluted}_i$  & $-$0.352$^{**}$ & $-$0.478$^{***}$ & $-$0.323$^{**}$ & $-$0.558$^{***}$ & $-$0.354$^{**}$ & $-$0.571$^{***}$ & $-$0.324$^{**}$ & $-$0.567$^{***}$ \\ 
  & (0.156) & (0.146) & (0.160) & (0.154) & (0.162) & (0.152) & (0.162) & (0.154) \\ 
 \hline \\[-1.8ex] 
City fixed effects & Yes & No & Yes & No & Yes & No & Yes & No \\ 
Industry fixed effects & Yes & No & Yes & No & Yes & No & Yes & No \\ 
Year fixed effects & Yes & No & Yes & No & Yes & No & Yes & No \\ 
City-year fixed effects & No & Yes & No & Yes & No & Yes & No & Yes \\ 
Industry-year fixed effects & No & Yes & No & Yes & No & Yes & No & Yes \\ 
City-industry fixed effects & No & Yes & No & Yes & No & Yes & No & Yes \\ 
Observations & 61,297 & 61,297 & 42,916 & 42,916 & 42,930 & 42,930 & 42,947 & 42,947 \\ 
R$^{2}$ & 0.432 & 0.878 & 0.440 & 0.879 & 0.443 & 0.882 & 0.439 & 0.879 \\ 
    \end{tabular}
    \end{adjustbox}
    \begin{tablenotes}
      \small
      \item 
      Note: $^{*}$ Significance at the 10\%, $^{**}$ Significance at the 5\%, $^{***}$ Significance at the 1\% Heteroskedasticity-robust standard errors in parentheses are clustered by industry
\end{tablenotes}
\end{table}




Our key assumption is that the effectiveness of the policy is lower in cities dominated by SOEs, which face a softer budget constraint. We expect, therefore, a lesser coefficient (smaller in absolute value, or insignificant) for those cities. To test this assumption, we compute for each city the SOEs' output share, share of capital, and share of employment. Then we split the sample into two sub-samples; SOEs dominated sub-sample (table \ref{table_5}, Panel A) is made of cities where the SOEs' output share (respectively capital and employment) is above the 60th decile\footnote{ Similar results hold for the 70th and 80th deciles, they are available upon request } of the total distribution, while non-SOEs sub-sample (table \ref{table_5}, Panel B) includes the remaining cities, below the 60th decile. 


Columns 3 and 4 report the estimates obtained when the 60th decile is based upon SOEs' industrial output share (resp. SOEs' capital share in columns 5-6 and SOEs' employment share in columns 7-8). The coefficient of interest remains negative and significant in the sub-sample of cities with a smaller presence of SOEs, below the 60th decile; It becomes insignificant in the sub-sample of cities with a stronger presence of SOEs, above the 60th decile. These findings confirm that the policy is attenuated in the polluting sectors where the presence of SOEs is large. SOEs can adopt business strategies less constrained by the new regulation than private firms. Because they receive financial and political support from the local government, they do not need to reduce, cut or relocate the production. SBC helps them to circumvent the policy and to absorb the costs linked to it. 

\subsection{Testing for parallel trends}
\addcontentsline{toc}{subsection}{Testing for parallel trends}

We must check that our strategy satisfies the parallel trends assumption, by showing that SO2 emissions trajectories are not deviated before the treatment (i.e., before the introduction of local environmental regulations). One might think for instance, that certain local governments anticipated the implementation of the environmental regulation and decided to enforce it before the treatment year. The test for the parallel trend assumption consists of replacing the treatment variable $\text { Period }$ with yearly dummies. The new specification becomes:

\begin{equation} \label{eq:equation_3}
\begin{aligned}
 \text {Log SO2 emission }_{i k t}= & \sum_{t=2002}^{2007} \alpha (Target_{i}  \times \text { Polluted sectors }_{k} \times year _{t}) + \\
 & \theta {X}_{i k t}+\nu_{ik}+\lambda_{it} +\phi_{kt} +\epsilon_{ikt}
 \end{aligned}
\end{equation}

In which $year_t$ is a dummy set equal to one whit $t$ ranging from years 2003 to 2007. 


The estimate of $\alpha$ captures the effect of the environmental policy on the log of SO2 emissions in the whole sample and in the subsamples of SOEs and non-SOEs cities before the policy was implemented. If the parallel trend assumption holds, the coefficient $\alpha$ should not be significant before 2006. Table \ref{table_6} reports the results. The coefficients are all insignificant at the usual levels before the treatment year, validating the parallel trend assumption. In the remaining columns, we split our sample between SOEs cities (column 3, 5 and 7) and non-SOEs cities (column 2, 4, 6). In the even columns, the coefficients for non-SOEs cities are negative, and significant from 2006 onward, suggesting the early effect of the policy immediately after its introduction. By contrast, the policy does not affect the cities where the presence of SOEs is large (odd columns). Estimates are obtained from specifications that control for output, fixed assets and employment.


\begin{table}[!htb] \centering
  \caption{\\ Test of parallel trend assumption} 
\label{table_6}
\begin{adjustbox}{width=\textwidth, totalheight=\textheight-2\baselineskip,keepaspectratio}
\begin{tabular}{@{\extracolsep{5pt}}lccccccc} 
\\[-1.8ex]\hline 
\hline \\[-1.8ex] 
 & \multicolumn{7}{c}{Estimate of $\alpha$, equation \ref{eq:equation_3} on page \pageref{eq:equation_3}} \\ 
\cline{2-8}
            
\\[-1.8ex]
            &\multicolumn{1}{c}{}&\multicolumn{2}{c}{Output}&\multicolumn{2}{c}{Capital}&\multicolumn{2}{c}{employment}\\
\\[-1.8ex] & (1) & (2) & (3) & (4) & (5) & (6) & (7)\\
 \\[-1.8ex]$year_t$ varying from: & Full sample & No SOE & SOE & No SOE & SOE & No SOE & SOE\\
 \hline \\[-1.8ex] 
  2003 & $-$0.110 & $-$0.202 & $-$0.797 & $-$0.171 & $-$0.547 & $-$0.187 & $-$0.341 \\ 
  & (0.243) & (0.254) & (0.687) & (0.249) & (0.686) & (0.249) & (0.616) \\ 
  2004 & $-$0.037 & $-$0.229 & $-$0.254 & $-$0.199 & 0.108 & $-$0.219 & 0.149 \\ 
  & (0.239) & (0.262) & (0.682) & (0.257) & (0.669) & (0.259) & (0.590) \\ 
  2005 & $-$0.344 &  $-$0.281  &  $-$0.279  &   $-$0.271  & $-$0.211 &  $-$0.274 &  $-$0.307 \\ 
  & (0.268) & (0.288) & (0.780) & (0.284) & (0.776) & (0.287) & (0.670) \\ 
  2006 & $-$0.665$^{**}$ & $-$0.908$^{***}$ & $-$0.346 & $-$0.875$^{***}$ & $-$0.049 & $-$0.880$^{***}$ & $-$0.407 \\ 
  & (0.276) & (0.296) & (0.805) & (0.289) & (0.801) & (0.295) & (0.718) \\ 
  2007 & $-$0.578$^{**}$ & $-$0.790$^{***}$ & $-$0.352 & $-$0.784$^{***}$ & $-$0.020 & $-$0.775$^{***}$ & $-$0.414 \\ 
  & (0.273) & (0.285) & (0.804) & (0.279) & (0.792) & (0.284) & (0.722) \\ 
%  \\[-1.8ex]
\hline \\[-1.8ex] 
\\[-1.8ex]Fixed effects: & \multicolumn{3}{c}{city-year, industry-year, city-industry}\\
Observations & 61,297 & 42,916 & 18,381 & 42,930 & 18,367 & 42,947 & 18,350 \\ 
R$^{2}$ & 0.878 & 0.879 & 0.892 & 0.882 & 0.887 & 0.879 & 0.893 \\ 
\hline 
\hline \\[-1.8ex] 
\end{tabular}
\end{adjustbox}
\begin{tablenotes} 
 \small 
 \item 
\footnotesize{
Due to limited space, only the coefficients of interest are presented. $^{*}$ Significance at the 10\%, $^{**}$ Significance at the 5\%, $^{***}$ Significance at the 1\%. Heteroscedasticity-robust standard errors in parentheses are clustered by industry 
}
 
\end{tablenotes}
\end{table}

\subsection{Diffusion channels at work under the Soft Budget Constraint} \label{diffusion}
\addcontentsline{toc}{subsection}{Diffusion channels at work under the Soft Budget Constraint}




Four diffusion channels are documented in this section. First, we look at TCZ cities, where we expect a stronger reaction to environmental regulation. We pay attention to other policies run by the central government such as Special Policies Zones (SPZ) and Go West policies, where the emphasis is put on the objective of growth. Second, we estimate Kuznets curves and validate the assumption that wealthier cities enjoy SO2 mitigation progress. Third, large firms are in a better position to negotiate with local authorities. Following this statement, we check whether the bargaining power translates into a weaker compliance with the environmental regulation. Fourth, we verify the Porter’s theory, according to which environmental regulation can be accommodated for through investment in green projects, which enhance productivity. The latter are more likely to be taken by private firms, as large corporations like SOEs tend to choose safer innovative projects. Beyond the SBC, we investigate to what extent SOEs response to environmental regulation is channeled through these four mechanisms. All reported results are obtained with city-year, industry-year, and city-industry fixed effects\footnote{Results with city, industry, and year fixed effects are available upon request}. 


\subsubsection{First mechanism: policies run in the cities}
\addcontentsline{toc}{subsubsection}{First mechanism: policies run in the cities}



Table \ref{table_7} provides evidence that our results are sensitive to the political incentives that cities are facing. We consider three categories of cities: TCZ cities, Special Policies Zone (SPZ) cities, and finally cities close to the sea. While pollution reduction is clearly set as the top political priority for TCZ cities, it can compete for SPZ cities and cities far away from the sea with other political aims. The latter include the 'Go West' policy, which refers to a strategy launched in 2000, when the Chinese government decided to boost the economy of the western areas by pouring billions of US dollars into infrastructure, roads, facilities, and improving the skills of the workers (\citealt{Chen2018-ki}). This strategy provides incentives to firms, and more particularly, SOEs to downsize the production in favor of the new (attractive) cities located in China's Western hinterlands. We consider $Coastal_i$ which is one if city $i$ is away from the hinterland and close to the sea, which has historically always been a very attractive area, and it is zero if the city is located in the western areas and concerned by the Go West policy. 


The other policy is called SPZ. It aims at boosting the attractiveness of SPZ cities for foreign firms, exporters, or high tech firms, which benefit from lower taxes, access to cheaper credits, or subsidies, among others (\citealt{Wang2008-gv,Hering2014-af})\footnote{ Those SPZ include High-technology Industry Development Areas, Economic and Technological Development Areas, and Export Processing Zones.}. We define $SPZ_i$ equal to one if city $i$ belongs to the SPZ. In our sample we have 60 SPZ cities, and 108 cities adjacent to the sea.
 

%\begin{sidewaystable}%[!htb] \centering
\begin{table}[!htb] \centering 
  \caption{\\ TCZ, Go West, SPZ policies and environmental regulation effectiveness} 
\label{table_7}
\begin{adjustbox}{width=\textwidth, totalheight=\textheight-2\baselineskip,keepaspectratio}
\begin{tabular}{@{\extracolsep{5pt}}lcccccc} 
\\[-1.8ex]\hline 
\hline \\[-1.8ex] 
 & \multicolumn{6}{c}{Dependent variable $\text { SO2 emission }_{i k t}$} \\ 
\cline{2-7}
            
\\[-1.8ex]
            &\multicolumn{1}{c}{TCZ}&\multicolumn{1}{c}{No TCZ}&\multicolumn{1}{c}{ Coastal }&\multicolumn{1}{c}{No  Coastal}&\multicolumn{1}{c}{ SPZ }&\multicolumn{1}{c}{No  SPZ}\\
\\[-1.8ex] &  (1) &  (2) &  (3) &  (4) &  (5) &  (6)\\ 
\hline \\[-1.8ex] 
   $target_c \times \text{Period} \times \text{Polluted}_i$  & $-$0.528$^{***}$ & $-$0.803 & $-$0.415$^{**}$ & $-$0.632$^{**}$ & $-$0.494$^{***}$ & $-$1.020$^{***}$ \\ 
  & (0.155) & (0.919) & (0.170) & (0.267) & (0.188) & (0.366) \\ 
\hline \\[-1.8ex] 
\\[-1.8ex]Fixed effects: & \multicolumn{3}{c}{city-year, industry-year, city-industry}\\
Observations & 43,684 & 17,613 & 33,662 & 27,635 & 28,078 & 28,760 \\ 
R$^{2}$ & 0.875 & 0.901 & 0.880 & 0.888 & 0.868 & 0.893 \\ 
\hline 
\hline \\[-1.8ex] 
\end{tabular}
\end{adjustbox}
\begin{tablenotes} 
 \small 
 \item
\footnotesize{
$^{*}$ Significance at the 10\%, $^{**}$ Significance at the 5\%, $^{***}$ Significance at the 1\%. Heteroscedasticity-robust standard errors in parentheses are clustered by industry 
}
 
\end{tablenotes}
\end{table}
%\end{sidewaystable}



Our coefficient of interest remains negative and significant at 1$\%$ for TCZ cities (column 1). It is no more significant for no TCZ cities (column 2). This result is important as it suggests that reduction mandates (Target) and TCZ policies (local and central level) are complementary.  Interestingly, the reaction of cities located in the hinterlands (No Coastal) to the regulation stands at -0.632, to be compared to -0.415 for Coastal areas. Finally, as expected, the trade-off between growth and pollution reduction is biased towards growth in SPZ cities, where the coefficient of interest is lower, in absolute value, than in non SPZ cities (-0.494 for SPZ versus -1.020 for No SPZ). These findings suggest the possible influence of the policies on the performance of the cities dominated by SOEs\footnote{The output share of SOEs in TCZ is 25.3\%, 29\% in the hinterlands, 26\% in SPZs, whereas the average output share of SOEs in the total sample reaches 24\%.}.


\subsubsection{Second mechanism: level of concentration} \label{concentration}
\addcontentsline{toc}{subsubsection}{Second mechanism: level of concentration}





In this section, we look at the effect of size and industrial concentration on environmental regulation effectiveness. Large corporations are more likely to select safer innovative projects, while green and risky projects are undertaken by small or private firms. Besides, large firms can influence local authorities concerning the effective enforcement of environmental regulation, which is also an expression of the SBC. If this is true, private firms should adjust more to the regulation than SOEs. 


Our indicator of industrial concentration is based upon an Herfindahl index, computed as the average of the sum of the squared market share of industry $k$ in city $i$ over the period 2002-2005. Its values are ranging in an interval of 0.017 to 0.82. We define (low) concentrated cities as referring to cities above (below) the 60th decile of this indicator. We re-run our equation of interest in the sub-samples of concentrated versus no concentrated cities. Results are reported in table \ref{table_8}. 




\begin{table}[!htb] \centering
  \caption{\\ Industrial concentration and environmental regulation effectiveness} 
\label{table_8}
\begin{adjustbox}{width=\textwidth, totalheight=\textheight-2\baselineskip,keepaspectratio}
\begin{tabular}{@{\extracolsep{5pt}}lcccc} 
\\[-1.8ex]\hline 
\hline \\[-1.8ex] 
 & \multicolumn{2}{c}{Dependent variable $\text { SO2 emission }_{i k t}$} \\ 
\cline{2-3}
            
\\[-1.8ex]
            &\multicolumn{1}{c}{Concentrated}&\multicolumn{1}{c}{No Concentrated}\\
\\[-1.8ex] & (1) & (2)\\ 
\hline \\[-1.8ex] 
   $target_c \times \text{Period} \times \text{Polluted}_i$  & $-$0.048 & $-$0.565$^{***}$ \\ 
  & (0.341) & (0.158) \\ 
 \hline \\[-1.8ex] 
\hline \\[-1.8ex] 
\\[-1.8ex]Fixed effects: & \multicolumn{2}{c}{city-year, industry-year, city-industry}\\
Observations & 18,303 & 42,994 \\ 
R$^{2}$ & 0.906 & 0.870 \\ 
\hline 
\hline \\[-1.8ex] 
\end{tabular}
\end{adjustbox}
\begin{tablenotes} 
 \small 
 \item
\footnotesize{$^{*}$ Significance at the 10\%, $^{**}$ Significance at the 5\%, $^{***}$ Significance at the 1\%. Heteroscedasticity-robust standard errors in parentheses are clustered by industry 
}
 
\end{tablenotes}
\end{table}



Results confirm that size and industrial concentration matter. In cities characterized by a low degree of industrial concentration, i.e., below the 60th decile of the Herfindahl index\footnote{ Similar results hold for the 70th and 80th deciles, they are available upon request }, polluted sectors are sensitive to environmental regulation. The coefficient of interest reaches -0.565 (table \ref{table_8}, column 2), it is highly significant at 1$\%$. In concentrated cities it falls to -0.048 (column 1) not significantly different from zero, suggesting that larger companies are in a stronger bargaining position and can impose their own objectives. But SOEs are large companies: their output share in concentrated cities reaches 32\%, while it is only 15\% in no concentrated cities. Part of the SBC  they face can be explained therefore by the degree of industrial concentration, which characterize the cities in which they operate. 

\subsubsection{Third mechanism: Kuznets curve}
\addcontentsline{toc}{subsubsection}{Third mechanism: Kuznets curve}



We address the concern that poorer cities could be less sensitive to the environmental regulation because of the correlation between the wealth of the inhabitants and the sensitivity to the environment. Empirical evidence about this correlation is extensive and widely reported in the Kuznets curve literature: wealthier households have the financial capacity to consume in line with their preferences for goods and services that protect the environment (\citealt{Berger2019-jl,Chen2018-ki}), or they can escape from polluted cities (\citealt{Chen2017-ro}).


We use equation \ref{eq:equation_4} below to estimate the relationship between a Chinese city's emission of SO2 and its characteristics, including log per capita income and squared log per capita income. Following the academic literature studying the environmental Kuznets curve, this allows us to test for whether there is a $``$turning point$"$ such that when a city's per capita income exceeds this turning point level, the association between economic growth and pollution becomes negative. In table \ref{table_9} reported below, we find that wealthier cities enjoy SO2 mitigation progress and that the key turning point ranges from US$\$$ 2214 to US$\$$ 3872\footnote{In \cite{Kahn2016-fi} the turning point is found to be equal to US$\$$ 10 000. It is computed from data on particulate matter annual mean concentration (PM10).}. 

\begin{equation} \label{eq:equation_4}
\begin{aligned}
\text {Log SO2 emission }_{i k t} = & \alpha \text{(ln gdp per cap)}_{ct}  + \beta\text{(ln gdp per cap)}^2_{ct}  + \gamma \text{(ln population)}_{ct} \\ 
& + \nu_{c}+\lambda_{k}+\phi_{t}+\epsilon_{i k t}
\end{aligned}
\end{equation}

Moreover, further analysis in the table \ref{table_9} documents that cities' characteristics – TCZ versus non-TCZ, concentrated versus no concentrated, and SOEs (no SOEs) dominated cities – matter for the existence of the environmental Kuznets curve. For non TCZ, cities with a high level of industrial concentration, and more importantly high (be it based upon output, capital or employment) share of SOEs, we are not able to detect a turning point, e.g., a level of per capita income, above which the relationship between local economic growth and pollution levels reverses and becomes negative. We notice also that the share of SOEs dominated cities above the turning point is 78\% (only 55\% for no SOEs dominated cities), and 22\% (45\%) below it (according to table \ref{table_9}, column 1). This suggests that most SOEs Dominated cities should react to the regulation by decreasing their emission of SO2, being on the portion of the curve where the relationships between pollution and economic growth is negative\footnote{Our data set documents that the median GDP per capita of SOEs dominated cities is higher by RMB 4 000, being RMB 24 730 for SOE dominated cites and RMB 20 175 for no SOEs dominated cities. The mean GDP per capita of no SOEs dominated cities and SOEs dominated cities are not significantly different: RMB 28 458 for the former and RMB 28 539 for the latter.}. The absence of reaction is specific to SOEs dominated cities and can be interpreted as symptomatic of the SBC.


\begin{sidewaystable}%[!htb] \centering
  \caption{\\ Environmental regulation effectiveness along the Kuznets curves} 
\label{table_9}
\begin{adjustbox}{width=\textwidth, totalheight=\textheight-2\baselineskip,keepaspectratio}
\begin{tabular}{@{\extracolsep{5pt}}lcccccccccc} 
\\[-1.8ex]\hline 
\hline \\[-1.8ex] 
 & \multicolumn{10}{c}{Dependent variable $\text { SO2 emission }_{i k t}$} \\ 
\cline{2-11}
            
\\[-1.8ex]
            &\multicolumn{2}{c}{City}&\multicolumn{2}{c}{Concentration}&\multicolumn{2}{c}{Output}&\multicolumn{2}{c}{Capital}&\multicolumn{2}{c}{Employment}\\
\\[-1.8ex] & (1) & (2) & (3) & (4) & (5) & (6) & (7) & (8) & (9) & (10)\\
 \\[-1.8ex]& TCZ & No TCZ & Concentrated & No Concentrated & SOE dominated & No SOEs dominated & SOE dominated & No SOEs dominated & SOE dominated & No SOEs dominated\\
 \hline \\[-1.8ex] 
  $\text{(ln gdp per cap)}_{ct}$  & 3.259$^{***}$ & 0.962 & 1.291 & 3.569$^{***}$ & 1.023 & 3.087$^{***}$ & 2.345 & 3.210$^{***}$ & 1.623 & 3.711$^{***}$ \\ 
  & (0.913) & (0.937) & (1.061) & (0.768) & (1.797) & (0.676) & (1.821) & (0.687) & (1.416) & (0.699) \\ 
   $\text{(ln gdp per cap) squared}_{ct}$  & $-$0.166$^{***}$ & $-$0.033 & $-$0.063 & $-$0.172$^{***}$ & $-$0.040 & $-$0.158$^{***}$ & $-$0.117 & $-$0.163$^{***}$ & $-$0.081 & $-$0.185$^{***}$ \\ 
  & (0.045) & (0.048) & (0.053) & (0.038) & (0.090) & (0.034) & (0.092) & (0.034) & (0.072) & (0.034) \\ 
   $\text{(ln population)}_{ct}$  & 0.266$^{***}$ & 0.300$^{**}$ & 0.434$^{**}$ & 0.261$^{***}$ & 0.372$^{*}$ & 0.215$^{**}$ & 0.286 & 0.216$^{**}$ & 0.080 & 0.242$^{***}$ \\ 
  & (0.094) & (0.138) & (0.186) & (0.087) & (0.194) & (0.087) & (0.184) & (0.086) & (0.191) & (0.086) \\ 
 \hline \\[-1.8ex] 
turning point RMB & 18661 & - & - & 31244 & - & 17864 & - & 18809 & - & 22467 \\ 
turning point Dollar & 2312 & - & - & 3872 & - & 2214 & - & 2331 & - & 2784 \\ 
SOEs cities above (below) & 
(78\%, 22\%)& - & - & 
(37\%, 63\%)& - & 
(78\%, 22\%)  & - & 
(77\%, 23\%)& - & 
(63\%, 37\%) \\
No SOEs cities above (below) & 
(55\%, 45\%) & - & - & 
(28\%, 72\%) & - & 
(55\%,45\%) & - &  
(55\%, 45\%) & - & 
(42\%, 58\%) \\
City fixed effects & Yes & Yes & Yes & Yes & Yes & Yes & Yes & Yes & Yes & Yes \\ 
Industry fixed effects & Yes & Yes & Yes & Yes & Yes & Yes & Yes & Yes & Yes & Yes \\ 
Year fixed effects & Yes & Yes & Yes & Yes & Yes & Yes & Yes & Yes & Yes & Yes \\ 
Observations & 42,570 & 17,520 & 18,175 & 41,915 & 18,336 & 41,754 & 18,300 & 41,790 & 18,271 & 41,819 \\ 
R$^{2}$ & 0.436 & 0.475 & 0.499 & 0.421 & 0.469 & 0.441 & 0.460 & 0.444 & 0.473 & 0.439 \\ 
\hline 
\hline \\[-1.8ex] 
\end{tabular}
\end{adjustbox}
\begin{tablenotes} 
 \small 
 \item
\footnotesize{
\textit{SOEs cities above (below)} indicates the share of SOEs dominated cities above and below the turning points, based on the \textit{GDP per capita} of 2007. 
\textit{No SOEs cities above (below)} refers to the share of cities above and below the turning points, whatever the ownership status, based on the \textit{GDP per capita} of 2007.%
Due to limited space, only the coefficients of interest are presented $^{*}$ Significance at the 10\%, $^{**}$ Significance at the 5\%, $^{***}$ Significance at the 1\%. Heteroscedasticity-robust standard errors in parentheses are clustered by industry 
}
 
\end{tablenotes}
\end{sidewaystable}

\subsubsection{Fourth mechanism: environmental regulation-induced TFP improvement}
\addcontentsline{toc}{subsubsection}{Fourth mechanism: environmental regulation-induced TFP improvement}



The evidence about the correlation between pollution abatement on one hand and productivity (scale economy and innovation) on the other hand is considerable, with potentially, a positive or negative sign. For a positive association, the rationale is the following: innovation aims at producing at a lower cost, allowing companies to use fewer inputs, and less dirty energy per unit of output. By imposing a lower strict limit for the emission of pollutants, the new regulation forces the firms to upgrade or leave the market (\citealt{Andersen2016-pa,Andersen2017-wf,Cole2008-pj}). This theory is also known as the Porter hypothesis (\citealt{Porter1995-vr}). However, the correlation can also be negative. According to the compliance cost theory, if the cost of environmental regulation impedes the improvement of productivity, it results in a decline in industrial performance. A recent paper, \cite{Yang2020-uw} show that the carbon emission trading system launched in 2017 verifies the Porter hypothesis in that it leads to an expansion of the employment scale and reduces the carbon emissions. 


However, these mechanisms may work only for private firms. There is a large body of literature showing that Chinese SOEs report lower economic performances (\citealt{Zhang2004-ij,Dougherty2007-qu,Qian1996-ab}) and lower TFP. Indeed the objective function does not focused on profit maximization, and the soft budget constraint implies that other emphases are put on competing objectives such as employment, social protection and incumbent protection, leaving aside productivity improvement. 

To disentangle these different assumptions, we estimate the following equation \ref{eq:equation_5}:

\begin{equation} \label{eq:equation_5}
TFP_{fikt}=\alpha\left(\text {Target}_{i} \times \text {Polluted sectors}_{k} \times \text { Period }\right)+ \zeta_{f}+ \nu_{i k}+\lambda_{i t}+\phi_{k t}+\epsilon_{i k t}
\end{equation}

Where the dependent variable $TFP_{fikt }$ is the firm $f$ productivity level computed with the Olley–Pakes algorithm (\citealt{Olley1996-yl}) at the firm-city-industry-time level. The panel structure of our data set allows us to address the endogeneity issues. First, the inclusion of city-time ($\lambda_{i t}$) fixed effects is particularly important because firms in a city faced with stronger regulatory requirement are more likely to be located in industrial areas prone to factors associated with citywide emission trends. Second, the inclusion of industry-time ($\phi_{k t}$) and city-industry ($\nu_{i k}$) fixed effects remove the trends among all firms in a particular industry that are unrelated to the environmental policy. Finally the inclusion of firms' fixed effects ($\zeta_{f}$) remove all unobserved factors contributing to a firm's TFP within a city and these effects are allowed to vary over time. 


Table \ref{table_10} reports the main coefficients of interest of equation \ref{eq:equation_5}, using firm level data over the period 2002-2007. Positive values of the coefficients imply that the target-based regulation led to an increase in the TFP variable, validating the Porter hypothesis according to which strict environmental regulation facilitates technological innovation, whereas negative values indicate that the cost of environmental protection faced by enterprises is harmful to investment in innovation and productivity improvement. 


Previous discussion has shown that the effect of the policy on SO2 emission is not homogeneous across cities depending on the status (TCZ versus non-TCZ), level of development (coastal and SPZ cities, cities below and above Kuznets turning points). Therefore, we control for this heterogeneity by distinguishing different sub-samples. In table \ref{table_10}, we compute the effects of the target-based policy on TFP for SOEs versus private firms. Panel A gathers firms belonging to TCZ cities (versus non-TCZ cities), and Panel B coastal (versus no-coastal). Panel C refers to firms belonging to cities where the level of industrial concentration is high as opposed to low. Finally, we run our model using different turning points from table \ref{table_9}, and the results are presented in a separate table \ref{table_11}. 




\begin{table}[!htb] \centering
  %\resizebox{1\textwidth}{!}{
    %\begin{threeparttable}
    \caption{\\ Reduction mandate - induced change in TFP}
      \begin{adjustbox}{width=\textwidth, totalheight=\textheight-2\baselineskip,keepaspectratio}
     \label{table_10}
      \begin{tabular}{@{\extracolsep{5pt}}lcccc}  
        \multicolumn{1}{l}{\textbf{Panel A: TCZ versus non-TCZ}} \\
        \toprule
        & \multicolumn{4}{c}{Dependent variable $\text { TFP }_{fikt}$} \\ 
\cline{2-5}
            
\\[-1.8ex]
            &\multicolumn{2}{c}{SOE}&\multicolumn{2}{c}{PRIVATE}\\
\\[-1.8ex] & (1) & (2) & (3) & (4)\\
 \\[-1.8ex]& TCZ & No TCZ & TCZ & No TCZ\\
 \hline \\[-1.8ex] 
   $target_c \times \text{Period} \times \text{Polluted}_i$  & 0.144$^{***}$ & $-$0.419 & $-$0.022 & $-$0.421$^{**}$ \\ 
  & (0.050) & (0.429) & (0.021) & (0.188) \\ 
 \hline \\[-1.8ex] 
Firm & Yes & Yes & Yes & Yes \\ 
City-industry &Yes & Yes & Yes & Yes \\ 
City-time & Yes & Yes & Yes & Yes \\ 
time-industry & Yes & Yes & Yes & Yes \\ 
Observations & 32,078 & 9,410 & 517,652 & 89,657 \\ 
R$^{2}$ & 0.953 & 0.961 & 0.861 & 0.869 \\

        \bottomrule
        \\ %%% Create second table
        \multicolumn{1}{l}{\textbf{Panel B: Coastal  versus non - Coastal}} \\
        \toprule
        & \multicolumn{4}{c}{Dependent variable $\text { TFP }_{fikt}$} \\ 
\cline{2-5}
            
\\[-1.8ex]
            &\multicolumn{2}{c}{SOE}&\multicolumn{2}{c}{PRIVATE}\\
\\[-1.8ex] & (1) & (2) & (3) & (4)\\
 \\[-1.8ex]&  $\text{Coastal}$  & $\text{No  Coastal}$  &  $\text{Coastal}$  & $\text{No  Coastal}$ \\
 \hline \\[-1.8ex] 
   $target_c \times \text{Period} \times \text{Polluted}_i$  & 0.158$^{**}$ & 0.119 & $-$0.012 & $-$0.087$^{**}$ \\ 
  & (0.063) & (0.098) & (0.023) & (0.036) \\ 
 \hline \\[-1.8ex] 
Firm & Yes & Yes & Yes & Yes \\ 
City-industry & Yes & Yes & Yes & Yes \\ 
City-time & Yes & Yes & Yes & Yes \\ 
time-industry & Yes & Yes & Yes & Yes \\ 
Observations & 19,540 & 21,948 & 477,084 & 130,225 \\ 
R$^{2}$ & 0.955 & 0.956 & 0.857 & 0.878 \\ 

\bottomrule 
\\ %%% Create second table
        \multicolumn{1}{l}{\textbf{Panel C: industrial concentration}} \\
        \toprule
         & \multicolumn{4}{c}{Dependent variable $\text { TFP }_{fikt}$} \\ 
\cline{2-5}
            
\\[-1.8ex]
            &\multicolumn{2}{c}{SOE}&\multicolumn{2}{c}{PRIVATE}\\
\\[-1.8ex] & (1) & (2) & (3) & (4)\\
 \\[-1.8ex]& Concentrated & No Concentrated & Concentrated & No Concentrated\\
 \hline \\[-1.8ex] 
   $target_c \times \text{Period} \times \text{Polluted}_i$  & 0.068 & 0.159$^{**}$ & $-$0.035 & $-$0.015 \\ 
  & (0.084) & (0.063) & (0.032) & (0.024) \\ 
 \hline \\[-1.8ex] 
Firm & Yes & Yes & Yes & Yes \\ 
City-industry & Yes & Yes & Yes & Yes \\ 
City-time & Yes & Yes & Yes & Yes \\ 
time-industry & Yes & Yes & Yes & Yes \\ 
Observations & 23,054 & 18,434 & 170,305 & 437,004 \\ 
R$^{2}$ & 0.957 & 0.953 & 0.869 & 0.859 \\ 
    \end{tabular}
    \end{adjustbox}
    \begin{tablenotes}
      \small
      \item 
      Note: $^{*}$p$<$0.1 $^{**}$p$<$0.05 $^{***}$p$<$0.01 \\
      Heteroskedasticity-robust standard errors in parentheses are clustered by industry
    \end{tablenotes}
\end{table}





Estimates are reported in tables \ref{table_10} and \ref{table_11}. In the sub-samples of SOEs firms located in TCZ cities, the coefficient of interest is 0.144, positive and significant at 1$\%$, confirming the Porter hypothesis. For SOEs in non-TCZ cities, and private firms in TCZ cities, it is not significant, therefore the regulation has no effect on technological improvement. Finally, it is negative and significant for private firms in non-TCZ cities, suggesting that for those firms, the cost of the policy impedes the improvement of productivity. Similar findings hold for coastal (no-coastal) areas, with SOEs (private) in coastal (non-coastal) areas being positively (adversely) affected by the environmental regulation: the coefficient for SOEs in coastal areas is set at 0.158, while for private firms in no-coastal areas, it is - 0.087. The level of concentration matters as well, as reflected by the coefficient for SOEs firms located in cities where we consider an Herfindahl index below the 60th decile: 0.159 which is significant at 1$\%$ \footnote{Similar results hold for the 70th and 80th deciles, they are available upon request}. It confirms that smaller firms are more likely to invest in greener technologies which are usually riskier. Table \ref{table_11} confirms that for SOEs located in cities where GDP per capita is sufficiently high (above the turning points), the demand for a better environment and for a cleaner model of production translates into a significant and positive reaction to the regulation. In other cities this result does not hold anymore. 


Overall the results suggest that the policy-induced technological improvement hold only for SOEs, located in TCZ cities, in wealthier cities, and to a lesser extent in cities where the level of industrial concentration is lower. Therefore, the weaker policy-induced decrease in pollution that is reported in Section \ref{analysis} for cities where the share of SOEs is higher does not seem to be driven by an intrinsically smaller effort in technological improvement. Finally, if the environmental policy-induced technological improvement and concomitant decrease in the emission of SO2 happens in the richest areas of the country, we cannot exclude this improvement to be due to companies adjusting to the regulation not only by improving their technology, as suggested by our calculus, but by physically (re)locating to provinces with lower environmental targets or weakest enforcement. The evidence about the pollution haven hypothesis in China is mixed: \cite{Wang2019-ju} do not support the pollution haven hypothesis in domestic trade during 2007–2012, while China seems to be $``$pollution heaven$"$ in South-South trade according to \cite{Lin2019-ft} or \cite{Sun2017-la}.

\begin{table}[!htb] \centering 
  \caption{\\ Reduction mandate - induced change in TFP: below and above turning points} 
\label{table_11}
\begin{adjustbox}{width=\textwidth, totalheight=\textheight-2\baselineskip,keepaspectratio}
\begin{tabular}{@{\extracolsep{5pt}}lccccc} 
\\[-1.8ex]\hline 
\hline \\[-1.8ex] 
 & \multicolumn{4}{c}{Dependent variable $\text { TFP }_{fikt}$} \\ 
\cline{2-6}
            
\\[-1.8ex]
            &\multicolumn{2}{c}{SOE}&\multicolumn{2}{c}{PRIVATE}\\
\\[-1.8ex] & (1) & (2) & (3) & (4) \\
 \\[-1.8ex]& Above & Below & Above & Below \\
 \hline \\[-1.8ex] 
$target_c \times \text{Period} \times \text{Polluted}_i$  &0.164$^{***}$ (0.059) & 0.101 (0.129) & $-$0.015 (0.021)& $-$0.126 (0.077) &  \\
Observations  & 12,359 & 28,393 & 337,534 & 263,001&  Column (4): $\text{No Concentrated}^a$\\
R$^{2}$  & 0.965 & 0.963 & 0.882 & 0.889&  RMB 31244\\
\hline 
$target_c \times \text{Period} \times \text{Polluted}_i$  &  0.136$^{**}$ (0.053) & 0.101 (0.144) & $-$0.017 (0.022) & $-$0.142(0.069)&  \\
Observations  & 20,996 & 19,756 & 449,304 & 151,231 &  Column (10): $\text{SOE No dominated}^a$ \\ 
R$^{2}$  &0.955 & 0.966 & 0.867 & 0.896&  RMB 22467\\
\hline 
$target_c \times \text{Period} \times \text{Polluted}_i$  & 0.133$^{**}$ (0.053) & $-$0.075 (0.339) & $-$0.018 (0.022)&  $-$0.291$^{*}$ (0.162)& \\
Observations  & 25,668 & 15,084 & 491,600 & 108,935 &  Column (8): $\text{SOE No dominated}^a$ \\ 
R$^{2}$  & 0.954 & 0.968 & 0.866 & 0.902&  RMB 18809\\
\hline 
$target_c \times \text{Period} \times \text{Polluted}_i$  & 
0.131$^{**}$ (0.053)& 0.005 (0.355)& $-$0.018  (0.022)& $-$0.342$^{**}$ (0.147) & \\
Observations  & 25,845 & 14,907 & 493,866 & 106,669 & Column (1): $TCZ^a$\\ 
R$^{2}$  &0.954 & 0.968 & 0.866 & 0.903&  RMB  18661\\ 
\hline 
$target_c \times \text{Period} \times \text{Polluted}_i$  &  0.131$^{***}$  (0.050) & 0.034 (0.355)  &$-$0.019 (0.023) & $-$0.332$^{**}$(0.153)& \\
Observations  & 26,739 & 14,013 & 502,319 & 98,216 &  Column (6): $\text{SOE No dominated}^a$ \\ 
R$^{2}$  & 0.954 & 0.969 & 0.865 & 0.904 &  RMB 17864\\
\hline \\[-1.8ex] 
\end{tabular}
\end{adjustbox}
\begin{tablenotes} 
 \small 
 \item \footnotesize{
The columns Above (Below) refer to firms in cities whose gdp per capita are strictly above (below) the Kuznets turning points. References for the latter are provided in the last column. \\
$a$ refers to the number of the column in table \ref{table_9}, which provides us with the estimated turning point. For instance 31 244 is estimated using the sub-sample of firms in no-concentrated cities, table \ref{table_9}, column 4. \\
Due to limited space, only the coefficients of interest are presented $^{*}$ Significance at the 10\%, $^{**}$ Significance at the 5\%, $^{***}$ Significance at the 1\%. Heteroscedasticity-robust standard errors in parentheses are clustered by industry 
}
\end{tablenotes}
\end{table}
\section{Conclusion} \label{conclusion} 
\addcontentsline{toc}{section}{Conclusion}



The concept of the SBC introduced by \cite{Kornai1993-kg} is a very fruitful concept that can be applied to a wide range of situations, beyond simple transition economics and economics of socialism. Vahabi \citeyear{Vahabi2001-bp, Vahabi2014-qy} summarizes these situations, which include many cases of soft budget constraints in market economies. This paper investigates one such situation, namely the case of SOEs, in reaction to the change in the environmental protection regime. This change consists in a switch from a top-down to a bottom-up approach in 2006 and a new emphasis put on local incentives and target-based policy. We compute the policy-induced reduction of SO2 emission at the city level and distinguish TCZ (no TCZ), rich (poor) areas, cities where the level of industrial concentration is below (above) a given threshold, and SOEs (no SOEs) dominated cities. The findings demonstrate that SOEs dominated cities did not decreased their SO2 emission in response to the environmental regulation. 


The empirical analysis is rooted in a unique and rich dataset provided by the Ministry of Environmental Protection (MEP) and by the State Environmental Protection Agency (SEPA), which collect the main data source of pollutants and wastes in China since 1980. The double difference in difference identification strategy allows us to quantify the effect of the environmental regulation on firms' emissions of pollution. 


Several mechanisms are at work to explain this absence of reaction of SOEs dominated cities. The influence of TCZ policy, the location along the Kuznets curve, and the degree of industrial concentration.  Given that the share of SOEs Dominated cities in TCZ cities is close to the sample average, and that SOEs dominated cities are richer, the first two mechanisms are not relevant.  On the other hand, SOEs dominated cities are characterized by a larger industrial concentration. As a consequence, they are in the position not to comply with the environmentally-induced budget constraint hardening, which does not constrain them. Last but not least, we scrutinize the policy-induced firms TFP improvement by controlling for the heterogeneity of the cities' responses to the environmental regulation, and we find that SOEs are improving their productivity to adjust to the environmental targets under certain circumstances, when they are located in TCZ, in relatively wealthier areas and above the Kuznets turning points. These results are robust to various specifications and inclusions of city-year, industry-year and city-industry fixed effects. The analysis of TFP is realized with the inclusion of firms fixed effects. Besides, we document a slightly negative effect of the environmental protection for certain private enterprises, which does not for the sub-samples of no TCZ and no Coastal cities, an outcome which may be unique to developing countries, as emphasized in the literature (see \cite{Jefferson2013-az}). Finally, we cannot exclude that companies can adjust to the regulation not only by improving their technology, but by physically (re)locating to the provinces with lower environmental targets or weaker enforcement.

\end{document}