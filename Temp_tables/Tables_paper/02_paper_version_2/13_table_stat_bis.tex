\documentclass[12pt]{article} 
\usepackage[utf8]{inputenc}
\usepackage{booktabs,caption,threeparttable, siunitx, adjustbox}

\begin{document}

\begin{table}[!htb] \centering
    \caption{\\ GDP per capita, population and SO2 emissions}
      \begin{adjustbox}{width=\textwidth, totalheight=\textheight-2\baselineskip,keepaspectratio}
    \label{table_3}
    \begin{tabular}{lrrrrrrrrrrrr}
      \multicolumn{1}{l}{\textbf{\small Panel A:}} \\
      \multicolumn{1}{l}{\textbf{\small GDP per capita and population}} \\
      \toprule
     & \multicolumn{3}{c}{No TCZ} & \multicolumn{3}{c}{TCZ} \\
      & (1)  & (2) & & (3)  & (4) \\
      & 2004-2005 &  2006-2007 & & 2004-2005 &2006-2007  \\
      
      \midrule
      $\text{gdp per capita}_i$    & 16,496 & 22,665 & & 23,992 & 32,984 \\
$\text{population}_i$ &     79 &     83 &  &  158 &    166  \\

      \bottomrule
      \\ %%% Create second table
        \multicolumn{1}{l}{\textbf{Panel B:}} \\
        \multicolumn{1}{l}{\textbf{SO2 emissions (millions of tonnes)}} \\
        \toprule
       & \multicolumn{3}{c}{No TCZ} & \multicolumn{3}{c}{TCZ} \\
      & (1)  & (2) & (2) - (1) & (4)  & (5)  &(5) - (4) \\
      & 2004-2005 &  2006-2007 & &  2004-2005 &2006-2007  \\
\hline \\[-1.8ex] 
Full sample &  2.624  & 2.833  &  0.209 &  9.736 &10.294  &  0.558  \\
Central     &  0.991  & 1.004  &  0.013 & 2.138 & 2.154  &   0.016  \\
Coastal     &  0.556  & 0.632  &  0.076 & 3.763 & 3.860 &   0.097  \\
Northeast   &  0.354  & 0.389  &  0.035 & 0.537 & 0.736 &    0.199  \\
Northwest   &  0.265  &  0.360 &  0.096 & 1.110 & 1.097 &    -0.012  \\
Southwest   &  0.459  & 0.448  &  -0.010 & 2.189 &  2.448 &   0.259  \\
      \bottomrule
      \hline
    \end{tabular}
    \end{adjustbox}
    \begin{tablenotes}
      \small
      \item 
      \footnotesize{
      Sources: Authors' own computation \\
  Panel A: $\text{gdp per capita}_i$ is in RMB and $\text{population}_i$ is in million. \\ $\text{gdp per capita}_i$ and and $\text{population}_i$ are averaged over 2004-2005 and 2006-2007. They are borrowed from the China City Statistical Yearbooks 2002–2007. \\
  Panel B: reported numbers are in millions of tonnes.  \\
  All variables are summed over the years 2004 and 2005 and over the years 2006 and 2007. \\
      (No) Dominated SOEs cities refer to cities where the output share of SOEs is (below) above a critical threshold, for instance the 60th decile.  \\
      }
    \end{tablenotes}
\end{table}

\end{document}