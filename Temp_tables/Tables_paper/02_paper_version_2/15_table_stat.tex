\documentclass[12pt]{article} 
\usepackage[utf8]{inputenc}
\usepackage{booktabs,caption,threeparttable, siunitx, adjustbox}

\begin{document}\begin{table}\centering
\caption{Summary statistics of Target by city characteristics}
\label{table_1}
\begin{adjustbox}{width=\textwidth, totalheight=\textheight-2\baselineskip,keepaspectratio}
\begin{tabular}{lrrr}
\toprule
{} & All Cities & No SOE dominated & SOE dominated \\
index                &            &                  &               \\
\midrule
Full sample          &     12.07\% &           11.89\% &        12.39\% \\
Central              &      8.97\% &            7.47\% &        10.99\% \\
Coastal              &     17.79\% &           18.04\% &        16.05\% \\
Northeast            &      5.76\% &            5.17\% &         6.81\% \\
Northwest            &      5.70\% &            2.87\% &         7.52\% \\
Southwest            &     16.83\% &           11.82\% &        21.53\% \\
Central              &      8.97\% &            7.47\% &        10.99\% \\
Eastern              &     14.13\% &           14.73\% &        11.52\% \\
Western              &     11.93\% &            8.61\% &        14.20\% \\
No TCZ               &      5.69\% &            5.03\% &         7.20\% \\
TCZ                  &     16.07\% &           16.75\% &        14.99\% \\
Concentrated city    &      9.72\% &            6.11\% &        12.87\% \\
No Concentrated city &     14.58\% &           15.34\% &        10.72\% \\
Coastal              &     17.79\% &           18.04\% &        16.05\% \\
No Coastal           &      9.64\% &            7.60\% &        11.83\% \\
\bottomrule
\end{tabular}
\end{adjustbox}
\begin{tablenotes} 
 \small 
 \item \\ 

Sources: Author's own computation 

The list of TCZ is provided by the State Council, 1998. 
(No) SOE Dominated cities refers to cities where the 
(output, capital, employment) share of SOEs is (below) above a critical threshold,
for instance the 6th decile

(No) Concentrated city refers to cities where the 
Herfhindal index is (below) above a critical threshold,
for instance the 6th decile
      
 
\end{tablenotes}
\end{table}

\end{document}