\documentclass[12pt]{article} 
\usepackage[utf8]{inputenc}
\usepackage{booktabs,caption,threeparttable, siunitx, adjustbox}

\begin{document}

\begin{table}[!htbp] \centering
    \caption{Summary statistics by city characteristics}
      \begin{adjustbox}{width=\textwidth, totalheight=\textheight-2\baselineskip,keepaspectratio}
    \label{tab:table2}
    \begin{tabular}{lrr}
      \multicolumn{1}{l}{\textbf{\small Panel A: City/city-industry}} \\
      \toprule
      & \multicolumn{1}{c}{No TCZ} & \multicolumn{1}{c}{TCZ} \\
      \midrule
      $SO2_{ikt}$  & 182,873 & 172,032 \\
      $\text{SO2 over population}_{ikt}$ &   2,754 &   1,633 \\
      $\text{gdp per capita}_i$ &  19,796 &  32,796 \\
      $\text{population}_i$ &      99 &     266 \\
      \bottomrule
      \\ %%% Create second table
      \multicolumn{1}{l}{\textbf{\small Panel B: SO2 pollution reduction}} \\
      \toprule
      {} & \multicolumn{2}{l}{\footnotesize difference (10.000 tons units)} \\
                  & No TCZ   & TCZ     \\
      Location    &          &         \\
      \midrule
      Full sample &    -192 & -436  \\
      Central     &    -238 & -232 \\
      Coastal     &    -138 & -659  \\
      Northeast   &    -206 &  -89 \\
      Northwest   &    -115 & -277  \\
      Southwest   &    -272 & -503 \\
      Non Coastal &    -215 & -302  \\
      Coastal     &    -140 & -578  \\
      \bottomrule
      \hline
    \end{tabular}
    \end{adjustbox}
    \begin{tablenotes}
      \small
      \item 
      Sources: Author's own computation \\
      Panel A provides summary statistics for the main variables used in the subsequent empirical analysis. \\
  Panel B provides summary statistics for the SO2 emission (in 10.000 tons) for Chinese cities (228), split into TCZ (140) and non-TCZ (88) cities \\
  $SO2_{ikt}$ is in kilos and $\text{population}_i$ in million
      \\
    \end{tablenotes}
\end{table}

\end{document}