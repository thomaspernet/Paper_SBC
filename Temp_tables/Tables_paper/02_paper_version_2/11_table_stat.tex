\documentclass[12pt]{article} 
\usepackage[utf8]{inputenc}
\usepackage{booktabs,caption,threeparttable, siunitx, adjustbox}

\begin{document}\begin{table}\centering
\caption{SO2 reduction during the subsequent FYPs}
\label{table_1}
\begin{adjustbox}{width=\textwidth, totalheight=\textheight-2\baselineskip,keepaspectratio}
\begin{tabular}{lrrrr}
\toprule
             Cities &  1998-2001 &  2002-2005 &  2006-2010 &  Target \\
\midrule
             No TCZ &     21.00\% &     64.00\% &    -11.00\% &  -6.00\% \\
                TCZ &     -7.00\% &     38.00\% &    -15.00\% & -16.00\% \\
 No Dominated SOE^ a &    -31.00\% &     30.00\% &    -16.00\% & -10.00\% \\
    Dominated SOE^ a &      6.00\% &     21.00\% &    -17.00\% & -12.00\% \\
        Full Sample &     -2.00\% &     45.00\% &    -13.00\% & -10.00\% \\
\bottomrule
\end{tabular}
\end{adjustbox}
\begin{tablenotes} 
 \small 
 \item \\ 

Sources: Author's own computation \ 

The list of TCZ is provided by the State Council, 1998.
"Official Reply to the State Council Concerning Acid Rain Control Areas
and Sulfur Dioxide Pollution Control Areas".
The information about the SO2 level are collected using various edition
of the China Environment Statistics Yearbook.
We compute the reduction of SO2 emission using the same methodology
as Chen and al.(2018). \ 

$a$ (No) Dominated SOEs cities refer to cities where the 
(output, capital, employment) share of SOEs is (below) above a critical threshold,
for instance the 6th decile
 
\end{tablenotes}
\end{table}

\end{document}