\documentclass[12pt]{article}
\usepackage[utf8]{inputenc}
\usepackage{booktabs,caption,threeparttable, siunitx, adjustbox}

\begin{document}

\begin{table}[!htbp] \centering
  \caption{SO2 reduction during the subsequent FYP}
  \begin{adjustbox}{width=\textwidth, totalheight=\textheight-2\baselineskip,keepaspectratio}
    \label{tab:table1}
    \begin{tabular}{lrrrr}
      \toprule
      {} & \multicolumn{2}{l}{Failure} & \multicolumn{1}{l}{Success} \\
      \hline
      &      (1998-2000) & (2001-2005) &  (2006-2010) \\
      \midrule
      \textbf{TCZ} & & & \\
      \text{\footnotesize{SO2 target}}       & -     & -     & -14\% &   \\
      \text{\footnotesize{SO2 \% reduction}} & -7\% & 38\%  & -15\%     &   \\

      \textbf{No TCZ} & & & \\
      \text{\footnotesize{SO2 target}}       & -     & -     & -4\%  &   \\
      \text{\footnotesize{SO2 \% reduction}} & 21\%  & 64\%  & -11\%     &   \\

      \textbf{All sample} & & & \\
      \text{\footnotesize{SO2 target}}       & -     & -10\% & -10\% &   \\
      \text{\footnotesize{SO2 \% reduction}} & -2\%     & 45\%  & -13\% &   \\

      \bottomrule
    \end{tabular}
    \end{adjustbox}
    \begin{tablenotes}
      \small
      \item 
      Sources: Author's own computation \\
      The list of TCZ is provided by the State Council, 1998. "Official Reply to the State Council Concerning Acid Rain Control Areas and Sulfur Dioxide Pollution Control Areas". The information about the SO2 level are collected using various edition of the China Environment Statistics Yearbook. We compute the reduction of SO2 emission using the same methodology as Chen and al.(2018). 
    \end{tablenotes}
\end{table}

\hfill \break

\begin{table}[!htbp] \centering
    \caption{ Summary Statistics by city-industry and city characteristics}
      \begin{adjustbox}{width=\textwidth, totalheight=\textheight-2\baselineskip,keepaspectratio}
    \label{tab:table2}
    \begin{tabular}{lrrrr}
      \multicolumn{1}{l}{\textbf{\small Panel A: City/city-industry}} \\
      \toprule
      & \multicolumn{2}{c}{No TCZ} & \multicolumn{2}{c}{TCZ} \\
      & \multicolumn{1}{c}{mean} & \multicolumn{1}{c}{std} & \multicolumn{1}{c}{mean} & \multicolumn{1}{c}{std}\\
      %{}                         & mean   & std    & mean   & std    \\
      \midrule
      SO2_{ikt}                  & 182,873 & 372,028 & 174,308 & 357,225 \\
      \text{count share SOE}_k   & 0.093  & 0.094  & 0.082  & 0.086  \\
      \text{output share SOE}_k  & 0.147  & 0.153  & 0.136  & 0.151  \\
      \text{capital share SOE}_k & 0.220  & 0.197  & 0.209  & 0.198  \\
      \text{labour share SOE}_k  & 0.188  & 0.162  & 0.176  & 0.162  \\
      \text{output}_{kit}        & 0.028  & 0.101  & 0.058  & 0.266  \\
      \text{capital}_{kit}       & 0.008  & 0.030  & 0.014  & 0.054  \\
      \text{labour}_{kit}        & 0.009  & 0.024  & 0.016  & 0.056  \\
      \bottomrule
      \\ %%% Create second table
      \multicolumn{1}{l}{\textbf{\small Panel B: SO2 emission by city-location}} \\
      \toprule
      {} & \multicolumn{2}{l}{\footnotesize difference (10.000 tons units)} & \multicolumn{2}{l}{variance (\footnotesize \%)} \\
      %& \multicolumn{1}{l}{No TCZ} & \multicolumn{1}{l}{TCZ} & %\multicolumn{1}{l}{No TCZ} & \multicolumn{1}{l}{TCZ}\\
                  & No TCZ   & TCZ      & No TCZ & TCZ  \\
      Location    &          &          &        &      \\
      \midrule
      Full sample &    -192 & -436 &     0.22 & 0.26 \\
      Central     &    -238 & -232 &     0.25 & 0.19 \\
      Coastal     &    -138 & -659 &     0.12 & 0.29 \\
      Northeast   &    -206 &  -89 &     0.30 & 0.08 \\
      Northwest   &    -115 & -277 &     0.21 & 0.31 \\
      Southwest   &    -272 & -503 &     0.31 & 0.29 \\
      Non Coastal &    -215 & -302 &     0.28 & 0.23 \\
      Coastal     &    -140 & -578 &     0.13 & 0.28 \\
      \bottomrule
      \hline
    \end{tabular}
    \end{adjustbox}
    \begin{tablenotes}
      \small
      \item 
      Sources: Author's own computation \\
      Panel A provides summary statistics for the main variables used in the subsequent empirical analysis. \\
  Panel B provides summary statistics for the SO2 emission (in 10.000 tons) for Chinese cities (228), split into TCZ (140) and non-TCZ (88) cities
      \\
    \end{tablenotes}
\end{table}
\end{document}
