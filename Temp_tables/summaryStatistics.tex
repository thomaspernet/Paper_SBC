\documentclass[12pt]{article}
\usepackage[utf8]{inputenc}
\usepackage{booktabs,caption,threeparttable, siunitx, adjustbox}

\begin{document}

\begin{table}[!htbp] \centering
  \caption{Summary of TCZ results over time}
  \begin{adjustbox}{width=\textwidth, totalheight=\textheight-2\baselineskip,keepaspectratio}
    \label{}
    \begin{tabular}{lrrrr}
      \toprule
      {} & \multicolumn{2}{l}{Failure} & \multicolumn{1}{l}{Success} \\
      \hline
      &      (1998-2000) & (2001-2005) &  (2006-2010) \\
      \midrule
      \textbf{TCZ} & & & \\
      \text{\footnotesize{SO2 target}}       & -     & -     & -14\% &   \\
      \text{\footnotesize{SO2 \% reduction}} & -7\% & 38\%  & X     &   \\

      \textbf{No TCZ} & & & \\
      \text{\footnotesize{SO2 target}}       & -     & -     & -4\%  &   \\
      \text{\footnotesize{SO2 \% reduction}} & 21\%  & 64\%  & X     &   \\

      \textbf{All sample} & & & \\
      \text{\footnotesize{SO2 target}}       & -     & -10\% & -10\% &   \\
      \text{\footnotesize{SO2 \% reduction}} & -2\%     & 45\%  & -14\% &   \\

      \bottomrule
    \end{tabular}
    \end{adjustbox}
    \begin{tablenotes}
      \small
      \item The TCZ policy is evaluated at 2 points in time, 2000 and 2010. In 2005, the chinese governement, facing a poor environmental achievement, decided to reinforce the environmental policy for the TCZ. The TCZ and the 11th FYP have now joint results, say differently, each TCZ city have a clear environmental objective. When both 11th FYP and TCZ have joint objectives, the environmental policy is a success. Prior years demonstrate a failure\\
      Sources: Author's own computation
    \end{tablenotes}
\end{table}

\hfill \break

\begin{table}[!htbp] \centering
    \caption{ Summary Statistics by city-industry and city characteristics}
      \begin{adjustbox}{width=\textwidth, totalheight=\textheight-2\baselineskip,keepaspectratio}
    \label{}
    \begin{tabular}{lrrrr}
      \multicolumn{1}{l}{\textbf{\small Panel A: City/city-industry}} \\
      \toprule
      & \multicolumn{2}{l}{No TCZ} & \multicolumn{2}{l}{TCZ} \\
      {}                         & mean   & std    & mean   & std    \\
      \midrule
      SO2_{ikt}                  & 182873 & 372028 & 174308 & 357225 \\
      \text{count share SOE}_k   & 0.093  & 0.094  & 0.082  & 0.086  \\
      \text{output share SOE}_k  & 0.147  & 0.153  & 0.136  & 0.151  \\
      \text{capital share SOE}_k & 0.220  & 0.197  & 0.209  & 0.198  \\
      \text{labour share SOE}_k  & 0.188  & 0.162  & 0.176  & 0.162  \\
      \text{output}_{kit}        & 0.028  & 0.101  & 0.058  & 0.266  \\
      \text{capital}_{kit}       & 0.008  & 0.030  & 0.014  & 0.054  \\
      \text{labour}_{kit}        & 0.009  & 0.024  & 0.016  & 0.056  \\
      \bottomrule
      \\ %%% Create second table
      \multicolumn{1}{l}{\textbf{\small Panel B: SO2 emission by city-location}} \\
      \toprule
      {} & \multicolumn{2}{l}{\footnotesize difference (10.000 tons units)} & \multicolumn{2}{l}{variance (\footnotesize \%)} \\
                  & No TCZ   & TCZ      & No TCZ & TCZ  \\
      Location    &          &          &        &      \\
      \midrule
      Full sample &    -192 & -436 &     0.22 & 0.26 \\
      Central     &    -238 & -232 &     0.25 & 0.19 \\
      Coastal     &    -138 & -659 &     0.12 & 0.29 \\
      Northeast   &    -206 &  -89 &     0.30 & 0.08 \\
      Northwest   &    -115 & -277 &     0.21 & 0.31 \\
      Southwest   &    -272 & -503 &     0.31 & 0.29 \\
      Non Coastal &    -215 & -302 &     0.28 & 0.23 \\
      Coastal     &    -140 & -578 &     0.13 & 0.28 \\
      \bottomrule
      \hline
    \end{tabular}
    \end{adjustbox}
    \begin{tablenotes}
      \small
      \item Panel A provides summary statistics for the main variables used in the subsequent empirical analysis. 
       \\
      Panel B provides Difference (variance) of the average SO2 emission (in 10.000 tons unit) before and after the introduction of the 11th FYP by TCZ and No TCZ. More precisely, we sum the SO2 emission by city for the period before 2005 and after 2005, then we computed the average across city. Next, we computed the difference between TCZ and No TCZ. First row shows that TCZ cities reduced the SO2 emission by 26\% while No TCZ by 22\% during the second period compared with the first period
      \\
      Sources: Author's own computation
    \end{tablenotes}
\end{table}
\end{document}
