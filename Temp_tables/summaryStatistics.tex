\documentclass[12pt]{article}
\usepackage[utf8]{inputenc}
\usepackage{booktabs,caption,threeparttable, siunitx, adjustbox}

\begin{document}


\begin{table}[!htbp] \centering 
    %\resizebox{1\textwidth}{!}{
    \begin{threeparttable} 
    \caption{\small Summary of TCZ results over time}
        \begin{tabular}{lrrrr}
        \toprule
            {} & \multicolumn{2}{l}{Failure} & \multicolumn{1}{l}{Success} \\
            \hline
             &      (1998-2000) & (2001-2005) &  (2006-2010) \\
            \midrule
            \textbf{TCZ} & & & \\
                \text{\footnotesize{SO2 target}} &-& -& -14\%& \\
                \text{\footnotesize{SO2 \% reduction}} & -31\%&26\% &X & \\
            
            \textbf{No TCZ} & & & \\
                \text{\footnotesize{SO2 target}} & -& -& -4\%& \\
                \text{\footnotesize{SO2 \% reduction}} & 34\% & 43\% &X & \\
            
            \textbf{All sample} & & & \\
                \text{\footnotesize{SO2 target}} & -& -10\%& -10\%& \\
                \text{\footnotesize{SO2 \% reduction}} & X& 34\%& -14\%& \\
            
        \bottomrule
        \end{tabular}
        \begin{tablenotes}
                \small
            \item The TCZ policy is evaluated at 2 points in time, 2000 and 2010. In 2005, the chinese governement, facing a poor environmental achievement, decided to reinforce the environmental policy for the TCZ. The TCZ and the 11th FYP have now joint results, say differently, each TCZ city have a clear environmental objective. When both 11th FYP and TCZ have joint objectives, the environmental policy is a success. Prior years demonstrate a failure\\
            Sources: Author's own computation 
        \end{tablenotes}
    \end{threeparttable}
    %}
\end{table} 

\hfill \break

\begin{table}[!htbp] \centering 
    %\resizebox{1\textwidth}{!}{
        \begin{threeparttable}  
            \caption{\small Summary Statistics by City-industry characteristics and City characteristics}
            \begin{tabular}{lrrrr}
            \multicolumn{1}{l}{\textbf{Panel A: City characteristics}} \\
            \toprule
                & \multicolumn{2}{l}{No TCZ} & \multicolumn{2}{l}{TCZ} \\
                {} &       mean &        std &       mean &        std \\
            \midrule
                SO2_{cit}                    & 182873 & 372028 & 174308 & 357225 \\
                \text{count share SOE}_i   &      0.093 &      0.094 &      0.082 &      0.086 \\
                \text{output share SOE}_i  &      0.147 &      0.153 &      0.136 &      0.151 \\
                \text{capital share SOE}_i &      0.220 &      0.197 &      0.209 &      0.198 \\
                \text{labour share SOE}_i  &      0.188 &      0.162 &      0.176 &      0.162 \\
                \text{output}_{cit}          &      0.028 &      0.101 &      0.058 &      0.266 \\
                \text{capital}_{cit}         &      0.008 &      0.030 &      0.014 &      0.054 \\
                \text{labour}_{cit}          &      0.009 &      0.024 &      0.016 &      0.056 \\
            \bottomrule
            \\ %%% Create second table
            \multicolumn{1}{l}{\textbf{Panel B: Count number of cities}} \\
            \toprule
                {} &  No TCZ & & TCZ & Total \\
                Full Sample     &      88 & &  140 &    228 \\
                SPZ     &      11 & &   48 &     59 \\
                Coastal &      27 & &   68 &     95 \\
            \hline
            \end{tabular}
            \begin{tablenotes}
                \small
                \item Panel A provides a summary statistics for the variables that vary by city-industry-year. Panel B counts the number of cities by TCZ and non TCZ. For instance, the full sample gathers 88 TCZ cities and 140 non TCZ cities. Among the 80 TCZ cities, 11 are also SPZ and 27 are located in the coastal areas. \\
                Sources: Author's own computation 
            \end{tablenotes}
            
\end{threeparttable}
%}
\end{table} 

\hfill \break

\begin{table}[!htbp] \centering 
    %\resizebox{1\textwidth}{!}{
    \begin{threeparttable} 
    \caption{\small Summary Statistics of the SO2 emission by City-location}
        \begin{tabular}{lrrrr}
        \toprule
            {} & \multicolumn{2}{l}{difference} & \multicolumn{2}{l}{variance} \\
             &      No TCZ &         TCZ &   No TCZ &  TCZ \\
            Location    &             &             &          &      \\
            \midrule
            Full sample & -1919200 & -4360643 &     0.22 & 0.26 \\
            Central     & -2381649 & -2318221 &     0.25 & 0.19 \\
            Coastal     & -1376522 & -6590904 &     0.12 & 0.29 \\
            Northeast   & -2060869 &  -893667 &     0.30 & 0.08 \\
            Northwest   & -1148382 & -2772765 &     0.21 & 0.31 \\
            Southwest   & -2717954 & -5025523 &     0.31 & 0.29 \\
            Non Coastal & -2147154 & -3015876 &     0.28 & 0.23 \\
            Coastal     & -1404192 & -5784514 &     0.13 & 0.28 \\
        \bottomrule
        \end{tabular}
        \begin{tablenotes}
                \small
            \item Difference (variance) of the average SO2 emission before and after the introduction of the 11th FYP by TCZ and No TCZ. More precisely, we sum the SO2 emission by city for the period before 2005 and after 2005, then we computed the average across city. Next, we computed the difference between TCZ and No TCZ. First row shows that TCZ cities reduced the SO2 emission by 26\% while No TCZ by 22\% \\
            Sources: Author's own computation 
        \end{tablenotes}
    \end{threeparttable}
    %}
\end{table} 
\end{document}

